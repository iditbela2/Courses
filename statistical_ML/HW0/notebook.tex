
% Default to the notebook output style

    


% Inherit from the specified cell style.




    
\documentclass[11pt]{article}

    
    
    \usepackage[T1]{fontenc}
    % Nicer default font (+ math font) than Computer Modern for most use cases
    \usepackage{mathpazo}

    % Basic figure setup, for now with no caption control since it's done
    % automatically by Pandoc (which extracts ![](path) syntax from Markdown).
    \usepackage{graphicx}
    % We will generate all images so they have a width \maxwidth. This means
    % that they will get their normal width if they fit onto the page, but
    % are scaled down if they would overflow the margins.
    \makeatletter
    \def\maxwidth{\ifdim\Gin@nat@width>\linewidth\linewidth
    \else\Gin@nat@width\fi}
    \makeatother
    \let\Oldincludegraphics\includegraphics
    % Set max figure width to be 80% of text width, for now hardcoded.
    \renewcommand{\includegraphics}[1]{\Oldincludegraphics[width=.8\maxwidth]{#1}}
    % Ensure that by default, figures have no caption (until we provide a
    % proper Figure object with a Caption API and a way to capture that
    % in the conversion process - todo).
    \usepackage{caption}
    \DeclareCaptionLabelFormat{nolabel}{}
    \captionsetup{labelformat=nolabel}

    \usepackage{adjustbox} % Used to constrain images to a maximum size 
    \usepackage{xcolor} % Allow colors to be defined
    \usepackage{enumerate} % Needed for markdown enumerations to work
    \usepackage{geometry} % Used to adjust the document margins
    \usepackage{amsmath} % Equations
    \usepackage{amssymb} % Equations
    \usepackage{textcomp} % defines textquotesingle
    % Hack from http://tex.stackexchange.com/a/47451/13684:
    \AtBeginDocument{%
        \def\PYZsq{\textquotesingle}% Upright quotes in Pygmentized code
    }
    \usepackage{upquote} % Upright quotes for verbatim code
    \usepackage{eurosym} % defines \euro
    \usepackage[mathletters]{ucs} % Extended unicode (utf-8) support
    \usepackage[utf8x]{inputenc} % Allow utf-8 characters in the tex document
    \usepackage{fancyvrb} % verbatim replacement that allows latex
    \usepackage{grffile} % extends the file name processing of package graphics 
                         % to support a larger range 
    % The hyperref package gives us a pdf with properly built
    % internal navigation ('pdf bookmarks' for the table of contents,
    % internal cross-reference links, web links for URLs, etc.)
    \usepackage{hyperref}
    \usepackage{longtable} % longtable support required by pandoc >1.10
    \usepackage{booktabs}  % table support for pandoc > 1.12.2
    \usepackage[inline]{enumitem} % IRkernel/repr support (it uses the enumerate* environment)
    \usepackage[normalem]{ulem} % ulem is needed to support strikethroughs (\sout)
                                % normalem makes italics be italics, not underlines
    

    
    
    % Colors for the hyperref package
    \definecolor{urlcolor}{rgb}{0,.145,.698}
    \definecolor{linkcolor}{rgb}{.71,0.21,0.01}
    \definecolor{citecolor}{rgb}{.12,.54,.11}

    % ANSI colors
    \definecolor{ansi-black}{HTML}{3E424D}
    \definecolor{ansi-black-intense}{HTML}{282C36}
    \definecolor{ansi-red}{HTML}{E75C58}
    \definecolor{ansi-red-intense}{HTML}{B22B31}
    \definecolor{ansi-green}{HTML}{00A250}
    \definecolor{ansi-green-intense}{HTML}{007427}
    \definecolor{ansi-yellow}{HTML}{DDB62B}
    \definecolor{ansi-yellow-intense}{HTML}{B27D12}
    \definecolor{ansi-blue}{HTML}{208FFB}
    \definecolor{ansi-blue-intense}{HTML}{0065CA}
    \definecolor{ansi-magenta}{HTML}{D160C4}
    \definecolor{ansi-magenta-intense}{HTML}{A03196}
    \definecolor{ansi-cyan}{HTML}{60C6C8}
    \definecolor{ansi-cyan-intense}{HTML}{258F8F}
    \definecolor{ansi-white}{HTML}{C5C1B4}
    \definecolor{ansi-white-intense}{HTML}{A1A6B2}

    % commands and environments needed by pandoc snippets
    % extracted from the output of `pandoc -s`
    \providecommand{\tightlist}{%
      \setlength{\itemsep}{0pt}\setlength{\parskip}{0pt}}
    \DefineVerbatimEnvironment{Highlighting}{Verbatim}{commandchars=\\\{\}}
    % Add ',fontsize=\small' for more characters per line
    \newenvironment{Shaded}{}{}
    \newcommand{\KeywordTok}[1]{\textcolor[rgb]{0.00,0.44,0.13}{\textbf{{#1}}}}
    \newcommand{\DataTypeTok}[1]{\textcolor[rgb]{0.56,0.13,0.00}{{#1}}}
    \newcommand{\DecValTok}[1]{\textcolor[rgb]{0.25,0.63,0.44}{{#1}}}
    \newcommand{\BaseNTok}[1]{\textcolor[rgb]{0.25,0.63,0.44}{{#1}}}
    \newcommand{\FloatTok}[1]{\textcolor[rgb]{0.25,0.63,0.44}{{#1}}}
    \newcommand{\CharTok}[1]{\textcolor[rgb]{0.25,0.44,0.63}{{#1}}}
    \newcommand{\StringTok}[1]{\textcolor[rgb]{0.25,0.44,0.63}{{#1}}}
    \newcommand{\CommentTok}[1]{\textcolor[rgb]{0.38,0.63,0.69}{\textit{{#1}}}}
    \newcommand{\OtherTok}[1]{\textcolor[rgb]{0.00,0.44,0.13}{{#1}}}
    \newcommand{\AlertTok}[1]{\textcolor[rgb]{1.00,0.00,0.00}{\textbf{{#1}}}}
    \newcommand{\FunctionTok}[1]{\textcolor[rgb]{0.02,0.16,0.49}{{#1}}}
    \newcommand{\RegionMarkerTok}[1]{{#1}}
    \newcommand{\ErrorTok}[1]{\textcolor[rgb]{1.00,0.00,0.00}{\textbf{{#1}}}}
    \newcommand{\NormalTok}[1]{{#1}}
    
    % Additional commands for more recent versions of Pandoc
    \newcommand{\ConstantTok}[1]{\textcolor[rgb]{0.53,0.00,0.00}{{#1}}}
    \newcommand{\SpecialCharTok}[1]{\textcolor[rgb]{0.25,0.44,0.63}{{#1}}}
    \newcommand{\VerbatimStringTok}[1]{\textcolor[rgb]{0.25,0.44,0.63}{{#1}}}
    \newcommand{\SpecialStringTok}[1]{\textcolor[rgb]{0.73,0.40,0.53}{{#1}}}
    \newcommand{\ImportTok}[1]{{#1}}
    \newcommand{\DocumentationTok}[1]{\textcolor[rgb]{0.73,0.13,0.13}{\textit{{#1}}}}
    \newcommand{\AnnotationTok}[1]{\textcolor[rgb]{0.38,0.63,0.69}{\textbf{\textit{{#1}}}}}
    \newcommand{\CommentVarTok}[1]{\textcolor[rgb]{0.38,0.63,0.69}{\textbf{\textit{{#1}}}}}
    \newcommand{\VariableTok}[1]{\textcolor[rgb]{0.10,0.09,0.49}{{#1}}}
    \newcommand{\ControlFlowTok}[1]{\textcolor[rgb]{0.00,0.44,0.13}{\textbf{{#1}}}}
    \newcommand{\OperatorTok}[1]{\textcolor[rgb]{0.40,0.40,0.40}{{#1}}}
    \newcommand{\BuiltInTok}[1]{{#1}}
    \newcommand{\ExtensionTok}[1]{{#1}}
    \newcommand{\PreprocessorTok}[1]{\textcolor[rgb]{0.74,0.48,0.00}{{#1}}}
    \newcommand{\AttributeTok}[1]{\textcolor[rgb]{0.49,0.56,0.16}{{#1}}}
    \newcommand{\InformationTok}[1]{\textcolor[rgb]{0.38,0.63,0.69}{\textbf{\textit{{#1}}}}}
    \newcommand{\WarningTok}[1]{\textcolor[rgb]{0.38,0.63,0.69}{\textbf{\textit{{#1}}}}}
    
    
    % Define a nice break command that doesn't care if a line doesn't already
    % exist.
    \def\br{\hspace*{\fill} \\* }
    % Math Jax compatability definitions
    \def\gt{>}
    \def\lt{<}
    % Document parameters
    \title{HW0}
    
    
    

    % Pygments definitions
    
\makeatletter
\def\PY@reset{\let\PY@it=\relax \let\PY@bf=\relax%
    \let\PY@ul=\relax \let\PY@tc=\relax%
    \let\PY@bc=\relax \let\PY@ff=\relax}
\def\PY@tok#1{\csname PY@tok@#1\endcsname}
\def\PY@toks#1+{\ifx\relax#1\empty\else%
    \PY@tok{#1}\expandafter\PY@toks\fi}
\def\PY@do#1{\PY@bc{\PY@tc{\PY@ul{%
    \PY@it{\PY@bf{\PY@ff{#1}}}}}}}
\def\PY#1#2{\PY@reset\PY@toks#1+\relax+\PY@do{#2}}

\expandafter\def\csname PY@tok@w\endcsname{\def\PY@tc##1{\textcolor[rgb]{0.73,0.73,0.73}{##1}}}
\expandafter\def\csname PY@tok@c\endcsname{\let\PY@it=\textit\def\PY@tc##1{\textcolor[rgb]{0.25,0.50,0.50}{##1}}}
\expandafter\def\csname PY@tok@cp\endcsname{\def\PY@tc##1{\textcolor[rgb]{0.74,0.48,0.00}{##1}}}
\expandafter\def\csname PY@tok@k\endcsname{\let\PY@bf=\textbf\def\PY@tc##1{\textcolor[rgb]{0.00,0.50,0.00}{##1}}}
\expandafter\def\csname PY@tok@kp\endcsname{\def\PY@tc##1{\textcolor[rgb]{0.00,0.50,0.00}{##1}}}
\expandafter\def\csname PY@tok@kt\endcsname{\def\PY@tc##1{\textcolor[rgb]{0.69,0.00,0.25}{##1}}}
\expandafter\def\csname PY@tok@o\endcsname{\def\PY@tc##1{\textcolor[rgb]{0.40,0.40,0.40}{##1}}}
\expandafter\def\csname PY@tok@ow\endcsname{\let\PY@bf=\textbf\def\PY@tc##1{\textcolor[rgb]{0.67,0.13,1.00}{##1}}}
\expandafter\def\csname PY@tok@nb\endcsname{\def\PY@tc##1{\textcolor[rgb]{0.00,0.50,0.00}{##1}}}
\expandafter\def\csname PY@tok@nf\endcsname{\def\PY@tc##1{\textcolor[rgb]{0.00,0.00,1.00}{##1}}}
\expandafter\def\csname PY@tok@nc\endcsname{\let\PY@bf=\textbf\def\PY@tc##1{\textcolor[rgb]{0.00,0.00,1.00}{##1}}}
\expandafter\def\csname PY@tok@nn\endcsname{\let\PY@bf=\textbf\def\PY@tc##1{\textcolor[rgb]{0.00,0.00,1.00}{##1}}}
\expandafter\def\csname PY@tok@ne\endcsname{\let\PY@bf=\textbf\def\PY@tc##1{\textcolor[rgb]{0.82,0.25,0.23}{##1}}}
\expandafter\def\csname PY@tok@nv\endcsname{\def\PY@tc##1{\textcolor[rgb]{0.10,0.09,0.49}{##1}}}
\expandafter\def\csname PY@tok@no\endcsname{\def\PY@tc##1{\textcolor[rgb]{0.53,0.00,0.00}{##1}}}
\expandafter\def\csname PY@tok@nl\endcsname{\def\PY@tc##1{\textcolor[rgb]{0.63,0.63,0.00}{##1}}}
\expandafter\def\csname PY@tok@ni\endcsname{\let\PY@bf=\textbf\def\PY@tc##1{\textcolor[rgb]{0.60,0.60,0.60}{##1}}}
\expandafter\def\csname PY@tok@na\endcsname{\def\PY@tc##1{\textcolor[rgb]{0.49,0.56,0.16}{##1}}}
\expandafter\def\csname PY@tok@nt\endcsname{\let\PY@bf=\textbf\def\PY@tc##1{\textcolor[rgb]{0.00,0.50,0.00}{##1}}}
\expandafter\def\csname PY@tok@nd\endcsname{\def\PY@tc##1{\textcolor[rgb]{0.67,0.13,1.00}{##1}}}
\expandafter\def\csname PY@tok@s\endcsname{\def\PY@tc##1{\textcolor[rgb]{0.73,0.13,0.13}{##1}}}
\expandafter\def\csname PY@tok@sd\endcsname{\let\PY@it=\textit\def\PY@tc##1{\textcolor[rgb]{0.73,0.13,0.13}{##1}}}
\expandafter\def\csname PY@tok@si\endcsname{\let\PY@bf=\textbf\def\PY@tc##1{\textcolor[rgb]{0.73,0.40,0.53}{##1}}}
\expandafter\def\csname PY@tok@se\endcsname{\let\PY@bf=\textbf\def\PY@tc##1{\textcolor[rgb]{0.73,0.40,0.13}{##1}}}
\expandafter\def\csname PY@tok@sr\endcsname{\def\PY@tc##1{\textcolor[rgb]{0.73,0.40,0.53}{##1}}}
\expandafter\def\csname PY@tok@ss\endcsname{\def\PY@tc##1{\textcolor[rgb]{0.10,0.09,0.49}{##1}}}
\expandafter\def\csname PY@tok@sx\endcsname{\def\PY@tc##1{\textcolor[rgb]{0.00,0.50,0.00}{##1}}}
\expandafter\def\csname PY@tok@m\endcsname{\def\PY@tc##1{\textcolor[rgb]{0.40,0.40,0.40}{##1}}}
\expandafter\def\csname PY@tok@gh\endcsname{\let\PY@bf=\textbf\def\PY@tc##1{\textcolor[rgb]{0.00,0.00,0.50}{##1}}}
\expandafter\def\csname PY@tok@gu\endcsname{\let\PY@bf=\textbf\def\PY@tc##1{\textcolor[rgb]{0.50,0.00,0.50}{##1}}}
\expandafter\def\csname PY@tok@gd\endcsname{\def\PY@tc##1{\textcolor[rgb]{0.63,0.00,0.00}{##1}}}
\expandafter\def\csname PY@tok@gi\endcsname{\def\PY@tc##1{\textcolor[rgb]{0.00,0.63,0.00}{##1}}}
\expandafter\def\csname PY@tok@gr\endcsname{\def\PY@tc##1{\textcolor[rgb]{1.00,0.00,0.00}{##1}}}
\expandafter\def\csname PY@tok@ge\endcsname{\let\PY@it=\textit}
\expandafter\def\csname PY@tok@gs\endcsname{\let\PY@bf=\textbf}
\expandafter\def\csname PY@tok@gp\endcsname{\let\PY@bf=\textbf\def\PY@tc##1{\textcolor[rgb]{0.00,0.00,0.50}{##1}}}
\expandafter\def\csname PY@tok@go\endcsname{\def\PY@tc##1{\textcolor[rgb]{0.53,0.53,0.53}{##1}}}
\expandafter\def\csname PY@tok@gt\endcsname{\def\PY@tc##1{\textcolor[rgb]{0.00,0.27,0.87}{##1}}}
\expandafter\def\csname PY@tok@err\endcsname{\def\PY@bc##1{\setlength{\fboxsep}{0pt}\fcolorbox[rgb]{1.00,0.00,0.00}{1,1,1}{\strut ##1}}}
\expandafter\def\csname PY@tok@kc\endcsname{\let\PY@bf=\textbf\def\PY@tc##1{\textcolor[rgb]{0.00,0.50,0.00}{##1}}}
\expandafter\def\csname PY@tok@kd\endcsname{\let\PY@bf=\textbf\def\PY@tc##1{\textcolor[rgb]{0.00,0.50,0.00}{##1}}}
\expandafter\def\csname PY@tok@kn\endcsname{\let\PY@bf=\textbf\def\PY@tc##1{\textcolor[rgb]{0.00,0.50,0.00}{##1}}}
\expandafter\def\csname PY@tok@kr\endcsname{\let\PY@bf=\textbf\def\PY@tc##1{\textcolor[rgb]{0.00,0.50,0.00}{##1}}}
\expandafter\def\csname PY@tok@bp\endcsname{\def\PY@tc##1{\textcolor[rgb]{0.00,0.50,0.00}{##1}}}
\expandafter\def\csname PY@tok@fm\endcsname{\def\PY@tc##1{\textcolor[rgb]{0.00,0.00,1.00}{##1}}}
\expandafter\def\csname PY@tok@vc\endcsname{\def\PY@tc##1{\textcolor[rgb]{0.10,0.09,0.49}{##1}}}
\expandafter\def\csname PY@tok@vg\endcsname{\def\PY@tc##1{\textcolor[rgb]{0.10,0.09,0.49}{##1}}}
\expandafter\def\csname PY@tok@vi\endcsname{\def\PY@tc##1{\textcolor[rgb]{0.10,0.09,0.49}{##1}}}
\expandafter\def\csname PY@tok@vm\endcsname{\def\PY@tc##1{\textcolor[rgb]{0.10,0.09,0.49}{##1}}}
\expandafter\def\csname PY@tok@sa\endcsname{\def\PY@tc##1{\textcolor[rgb]{0.73,0.13,0.13}{##1}}}
\expandafter\def\csname PY@tok@sb\endcsname{\def\PY@tc##1{\textcolor[rgb]{0.73,0.13,0.13}{##1}}}
\expandafter\def\csname PY@tok@sc\endcsname{\def\PY@tc##1{\textcolor[rgb]{0.73,0.13,0.13}{##1}}}
\expandafter\def\csname PY@tok@dl\endcsname{\def\PY@tc##1{\textcolor[rgb]{0.73,0.13,0.13}{##1}}}
\expandafter\def\csname PY@tok@s2\endcsname{\def\PY@tc##1{\textcolor[rgb]{0.73,0.13,0.13}{##1}}}
\expandafter\def\csname PY@tok@sh\endcsname{\def\PY@tc##1{\textcolor[rgb]{0.73,0.13,0.13}{##1}}}
\expandafter\def\csname PY@tok@s1\endcsname{\def\PY@tc##1{\textcolor[rgb]{0.73,0.13,0.13}{##1}}}
\expandafter\def\csname PY@tok@mb\endcsname{\def\PY@tc##1{\textcolor[rgb]{0.40,0.40,0.40}{##1}}}
\expandafter\def\csname PY@tok@mf\endcsname{\def\PY@tc##1{\textcolor[rgb]{0.40,0.40,0.40}{##1}}}
\expandafter\def\csname PY@tok@mh\endcsname{\def\PY@tc##1{\textcolor[rgb]{0.40,0.40,0.40}{##1}}}
\expandafter\def\csname PY@tok@mi\endcsname{\def\PY@tc##1{\textcolor[rgb]{0.40,0.40,0.40}{##1}}}
\expandafter\def\csname PY@tok@il\endcsname{\def\PY@tc##1{\textcolor[rgb]{0.40,0.40,0.40}{##1}}}
\expandafter\def\csname PY@tok@mo\endcsname{\def\PY@tc##1{\textcolor[rgb]{0.40,0.40,0.40}{##1}}}
\expandafter\def\csname PY@tok@ch\endcsname{\let\PY@it=\textit\def\PY@tc##1{\textcolor[rgb]{0.25,0.50,0.50}{##1}}}
\expandafter\def\csname PY@tok@cm\endcsname{\let\PY@it=\textit\def\PY@tc##1{\textcolor[rgb]{0.25,0.50,0.50}{##1}}}
\expandafter\def\csname PY@tok@cpf\endcsname{\let\PY@it=\textit\def\PY@tc##1{\textcolor[rgb]{0.25,0.50,0.50}{##1}}}
\expandafter\def\csname PY@tok@c1\endcsname{\let\PY@it=\textit\def\PY@tc##1{\textcolor[rgb]{0.25,0.50,0.50}{##1}}}
\expandafter\def\csname PY@tok@cs\endcsname{\let\PY@it=\textit\def\PY@tc##1{\textcolor[rgb]{0.25,0.50,0.50}{##1}}}

\def\PYZbs{\char`\\}
\def\PYZus{\char`\_}
\def\PYZob{\char`\{}
\def\PYZcb{\char`\}}
\def\PYZca{\char`\^}
\def\PYZam{\char`\&}
\def\PYZlt{\char`\<}
\def\PYZgt{\char`\>}
\def\PYZsh{\char`\#}
\def\PYZpc{\char`\%}
\def\PYZdl{\char`\$}
\def\PYZhy{\char`\-}
\def\PYZsq{\char`\'}
\def\PYZdq{\char`\"}
\def\PYZti{\char`\~}
% for compatibility with earlier versions
\def\PYZat{@}
\def\PYZlb{[}
\def\PYZrb{]}
\makeatother


    % Exact colors from NB
    \definecolor{incolor}{rgb}{0.0, 0.0, 0.5}
    \definecolor{outcolor}{rgb}{0.545, 0.0, 0.0}



    
    % Prevent overflowing lines due to hard-to-break entities
    \sloppy 
    % Setup hyperref package
    \hypersetup{
      breaklinks=true,  % so long urls are correctly broken across lines
      colorlinks=true,
      urlcolor=urlcolor,
      linkcolor=linkcolor,
      citecolor=citecolor,
      }
    % Slightly bigger margins than the latex defaults
    
    \geometry{verbose,tmargin=1in,bmargin=1in,lmargin=1in,rmargin=1in}
    
    

    \begin{document}
    
    
    \maketitle
    
    

    
    \hypertarget{statistical-learning---hw0}{%
\section{Statistical learning - HW0}\label{statistical-learning---hw0}}

    \hypertarget{pandas-data-analysis-in-python}{%
\section{Pandas: data analysis in
python}\label{pandas-data-analysis-in-python}}

For data-intensive work in Python the
\href{http://pandas.pydata.org}{Pandas} library has become essential.

What is \texttt{pandas}?

\begin{itemize}
\tightlist
\item
  Pandas can be thought of as \emph{NumPy arrays with labels} for rows
  and columns, and better support for heterogeneous data types, but it's
  also much, much more than that.
\item
  Pandas can also be thought of as \texttt{R}'s \texttt{data.frame} in
  Python.
\item
  Powerful for working with missing data, working with time series data,
  for reading and writing your data, for reshaping, grouping, merging
  your data, \ldots{}
\end{itemize}

Documentation: http://pandas.pydata.org/pandas-docs/stable/

\textbf{When do you need pandas?}

When working with \textbf{tabular or structured data} (like R dataframe,
SQL table, Excel spreadsheet, \ldots{}):

\begin{itemize}
\tightlist
\item
  Import data
\item
  Clean up messy data
\item
  Explore data, gain insight into data
\item
  Process and prepare your data for analysis
\item
  Analyse your data (together with scikit-learn, statsmodels, \ldots{})
\end{itemize}

ATTENTION!:

Pandas is great for working with heterogeneous and tabular 1D/2D data,
but not all types of data fit in such structures!

When working with array data (e.g.~images, numerical algorithms): just
stick with numpy

When working with multidimensional labeled data (e.g.~climate data):
have a look at

    \begin{Verbatim}[commandchars=\\\{\}]
{\color{incolor}In [{\color{incolor}329}]:} \PY{o}{\PYZpc{}}\PY{k}{matplotlib} inline
          
          \PY{k+kn}{import} \PY{n+nn}{numpy} \PY{k}{as} \PY{n+nn}{np}
          \PY{k+kn}{import} \PY{n+nn}{pandas} \PY{k}{as} \PY{n+nn}{pd}
          \PY{k+kn}{import} \PY{n+nn}{matplotlib}\PY{n+nn}{.}\PY{n+nn}{pyplot} \PY{k}{as} \PY{n+nn}{plt}
          
          \PY{n}{pd}\PY{o}{.}\PY{n}{options}\PY{o}{.}\PY{n}{display}\PY{o}{.}\PY{n}{max\PYZus{}rows} \PY{o}{=} \PY{l+m+mi}{8}
\end{Verbatim}


    \hypertarget{lets-start-with-a-showcase}{%
\subsection{Let's start with a
showcase}\label{lets-start-with-a-showcase}}

\hypertarget{case-1-titanic-data}{%
\paragraph{Case 1: Titanic data}\label{case-1-titanic-data}}

    \begin{Verbatim}[commandchars=\\\{\}]
{\color{incolor}In [{\color{incolor}330}]:} \PY{n}{df} \PY{o}{=} \PY{n}{pd}\PY{o}{.}\PY{n}{read\PYZus{}csv}\PY{p}{(}\PY{l+s+s1}{\PYZsq{}}\PY{l+s+s1}{titanic\PYZhy{}train.csv}\PY{l+s+s1}{\PYZsq{}}\PY{p}{)}
\end{Verbatim}


    \begin{Verbatim}[commandchars=\\\{\}]
{\color{incolor}In [{\color{incolor}331}]:} \PY{n}{df}\PY{o}{.}\PY{n}{head}\PY{p}{(}\PY{p}{)}
\end{Verbatim}


\begin{Verbatim}[commandchars=\\\{\}]
{\color{outcolor}Out[{\color{outcolor}331}]:}    PassengerId  Survived  Pclass  \textbackslash{}
          0            1         0       3   
          1            2         1       1   
          2            3         1       3   
          3            4         1       1   
          4            5         0       3   
          
                                                          Name     Sex   Age  SibSp  \textbackslash{}
          0                            Braund, Mr. Owen Harris    male  22.0      1   
          1  Cumings, Mrs. John Bradley (Florence Briggs Th{\ldots}  female  38.0      1   
          2                             Heikkinen, Miss. Laina  female  26.0      0   
          3       Futrelle, Mrs. Jacques Heath (Lily May Peel)  female  35.0      1   
          4                           Allen, Mr. William Henry    male  35.0      0   
          
             Parch            Ticket     Fare Cabin Embarked  
          0      0         A/5 21171   7.2500   NaN        S  
          1      0          PC 17599  71.2833   C85        C  
          2      0  STON/O2. 3101282   7.9250   NaN        S  
          3      0            113803  53.1000  C123        S  
          4      0            373450   8.0500   NaN        S  
\end{Verbatim}
            
    \textbf{What is the age distribution of the passengers?}

    \begin{Verbatim}[commandchars=\\\{\}]
{\color{incolor}In [{\color{incolor}332}]:} \PY{n}{df}\PY{p}{[}\PY{l+s+s1}{\PYZsq{}}\PY{l+s+s1}{Age}\PY{l+s+s1}{\PYZsq{}}\PY{p}{]}\PY{o}{.}\PY{n}{hist}\PY{p}{(}\PY{p}{)}
\end{Verbatim}


\begin{Verbatim}[commandchars=\\\{\}]
{\color{outcolor}Out[{\color{outcolor}332}]:} <matplotlib.axes.\_subplots.AxesSubplot at 0x111df2400>
\end{Verbatim}
            
    \begin{center}
    \adjustimage{max size={0.9\linewidth}{0.9\paperheight}}{output_7_1.png}
    \end{center}
    { \hspace*{\fill} \\}
    
    \begin{Verbatim}[commandchars=\\\{\}]
{\color{incolor}In [{\color{incolor}6}]:} \PY{n}{df}\PY{o}{.}\PY{n}{info}\PY{p}{(}\PY{p}{)}
\end{Verbatim}


    \begin{Verbatim}[commandchars=\\\{\}]
<class 'pandas.core.frame.DataFrame'>
RangeIndex: 891 entries, 0 to 890
Data columns (total 12 columns):
PassengerId    891 non-null int64
Survived       891 non-null int64
Pclass         891 non-null int64
Name           891 non-null object
Sex            891 non-null object
Age            714 non-null float64
SibSp          891 non-null int64
Parch          891 non-null int64
Ticket         891 non-null object
Fare           891 non-null float64
Cabin          204 non-null object
Embarked       889 non-null object
dtypes: float64(2), int64(5), object(5)
memory usage: 83.6+ KB

    \end{Verbatim}

    \textbf{How does the survival rate of the passengers differ between
sexes?}

    lambda is a short function we can use without defining it first. it can
be useful with pandas operation.

    \begin{Verbatim}[commandchars=\\\{\}]
{\color{incolor}In [{\color{incolor}11}]:} \PY{n}{df}\PY{o}{.}\PY{n}{groupby}\PY{p}{(}\PY{l+s+s1}{\PYZsq{}}\PY{l+s+s1}{Sex}\PY{l+s+s1}{\PYZsq{}}\PY{p}{)}\PY{p}{[}\PY{p}{[}\PY{l+s+s1}{\PYZsq{}}\PY{l+s+s1}{Survived}\PY{l+s+s1}{\PYZsq{}}\PY{p}{]}\PY{p}{]}\PY{o}{.}\PY{n}{aggregate}\PY{p}{(}\PY{k}{lambda} \PY{n}{x}\PY{p}{:} \PY{n}{x}\PY{o}{.}\PY{n}{sum}\PY{p}{(}\PY{p}{)} \PY{o}{/} \PY{n+nb}{len}\PY{p}{(}\PY{n}{x}\PY{p}{)}\PY{p}{)}
\end{Verbatim}


\begin{Verbatim}[commandchars=\\\{\}]
{\color{outcolor}Out[{\color{outcolor}11}]:}         Survived
         Sex             
         female  0.742038
         male    0.188908
\end{Verbatim}
            
    \textbf{Or how does it differ between the different classes?}

    \begin{Verbatim}[commandchars=\\\{\}]
{\color{incolor}In [{\color{incolor}12}]:} \PY{n}{df}\PY{o}{.}\PY{n}{groupby}\PY{p}{(}\PY{l+s+s1}{\PYZsq{}}\PY{l+s+s1}{Pclass}\PY{l+s+s1}{\PYZsq{}}\PY{p}{)}\PY{p}{[}\PY{l+s+s1}{\PYZsq{}}\PY{l+s+s1}{Survived}\PY{l+s+s1}{\PYZsq{}}\PY{p}{]}\PY{o}{.}\PY{n}{aggregate}\PY{p}{(}\PY{k}{lambda} \PY{n}{x}\PY{p}{:} \PY{n}{x}\PY{o}{.}\PY{n}{sum}\PY{p}{(}\PY{p}{)} \PY{o}{/} \PY{n+nb}{len}\PY{p}{(}\PY{n}{x}\PY{p}{)}\PY{p}{)}\PY{o}{.}\PY{n}{plot}\PY{p}{(}\PY{n}{kind}\PY{o}{=}\PY{l+s+s1}{\PYZsq{}}\PY{l+s+s1}{bar}\PY{l+s+s1}{\PYZsq{}}\PY{p}{)}
\end{Verbatim}


\begin{Verbatim}[commandchars=\\\{\}]
{\color{outcolor}Out[{\color{outcolor}12}]:} <matplotlib.axes.\_subplots.AxesSubplot at 0x110706e48>
\end{Verbatim}
            
    \begin{center}
    \adjustimage{max size={0.9\linewidth}{0.9\paperheight}}{output_13_1.png}
    \end{center}
    { \hspace*{\fill} \\}
    
    \hypertarget{the-pandas-data-structures-dataframe-and-series}{%
\section{\texorpdfstring{The pandas data structures: \texttt{DataFrame}
and
\texttt{Series}}{The pandas data structures: DataFrame and Series}}\label{the-pandas-data-structures-dataframe-and-series}}

A \texttt{DataFrame} is a \textbf{tablular data structure}
(multi-dimensional object to hold labeled data) comprised of rows and
columns, akin to a spreadsheet, database table, or R's data.frame
object. You can think of it as multiple Series object which share the
same index.

    \begin{Verbatim}[commandchars=\\\{\}]
{\color{incolor}In [{\color{incolor}13}]:} \PY{n}{df}
\end{Verbatim}


\begin{Verbatim}[commandchars=\\\{\}]
{\color{outcolor}Out[{\color{outcolor}13}]:}      PassengerId  Survived  Pclass  \textbackslash{}
         0              1         0       3   
         1              2         1       1   
         2              3         1       3   
         3              4         1       1   
         ..           {\ldots}       {\ldots}     {\ldots}   
         887          888         1       1   
         888          889         0       3   
         889          890         1       1   
         890          891         0       3   
         
                                                           Name     Sex   Age  SibSp  \textbackslash{}
         0                              Braund, Mr. Owen Harris    male  22.0      1   
         1    Cumings, Mrs. John Bradley (Florence Briggs Th{\ldots}  female  38.0      1   
         2                               Heikkinen, Miss. Laina  female  26.0      0   
         3         Futrelle, Mrs. Jacques Heath (Lily May Peel)  female  35.0      1   
         ..                                                 {\ldots}     {\ldots}   {\ldots}    {\ldots}   
         887                       Graham, Miss. Margaret Edith  female  19.0      0   
         888           Johnston, Miss. Catherine Helen "Carrie"  female   NaN      1   
         889                              Behr, Mr. Karl Howell    male  26.0      0   
         890                                Dooley, Mr. Patrick    male  32.0      0   
         
              Parch            Ticket     Fare Cabin Embarked  
         0        0         A/5 21171   7.2500   NaN        S  
         1        0          PC 17599  71.2833   C85        C  
         2        0  STON/O2. 3101282   7.9250   NaN        S  
         3        0            113803  53.1000  C123        S  
         ..     {\ldots}               {\ldots}      {\ldots}   {\ldots}      {\ldots}  
         887      0            112053  30.0000   B42        S  
         888      2        W./C. 6607  23.4500   NaN        S  
         889      0            111369  30.0000  C148        C  
         890      0            370376   7.7500   NaN        Q  
         
         [891 rows x 12 columns]
\end{Verbatim}
            
    \hypertarget{attributes-of-the-dataframe}{%
\subsubsection{Attributes of the
DataFrame}\label{attributes-of-the-dataframe}}

A DataFrame has besides a \texttt{index} attribute, also a
\texttt{columns} attribute:

    \begin{Verbatim}[commandchars=\\\{\}]
{\color{incolor}In [{\color{incolor}14}]:} \PY{n}{df}\PY{o}{.}\PY{n}{index}
\end{Verbatim}


\begin{Verbatim}[commandchars=\\\{\}]
{\color{outcolor}Out[{\color{outcolor}14}]:} RangeIndex(start=0, stop=891, step=1)
\end{Verbatim}
            
    \begin{Verbatim}[commandchars=\\\{\}]
{\color{incolor}In [{\color{incolor}15}]:} \PY{n}{df}\PY{o}{.}\PY{n}{columns}
\end{Verbatim}


\begin{Verbatim}[commandchars=\\\{\}]
{\color{outcolor}Out[{\color{outcolor}15}]:} Index(['PassengerId', 'Survived', 'Pclass', 'Name', 'Sex', 'Age', 'SibSp',
                'Parch', 'Ticket', 'Fare', 'Cabin', 'Embarked'],
               dtype='object')
\end{Verbatim}
            
    To check the data types of the different columns:

    \begin{Verbatim}[commandchars=\\\{\}]
{\color{incolor}In [{\color{incolor}16}]:} \PY{n}{df}\PY{o}{.}\PY{n}{dtypes}
\end{Verbatim}


\begin{Verbatim}[commandchars=\\\{\}]
{\color{outcolor}Out[{\color{outcolor}16}]:} PassengerId      int64
         Survived         int64
         Pclass           int64
         Name            object
                         {\ldots}   
         Ticket          object
         Fare           float64
         Cabin           object
         Embarked        object
         Length: 12, dtype: object
\end{Verbatim}
            
    An overview of that information can be given with the \texttt{info()}
method:

    \begin{Verbatim}[commandchars=\\\{\}]
{\color{incolor}In [{\color{incolor}17}]:} \PY{n}{df}\PY{o}{.}\PY{n}{info}\PY{p}{(}\PY{p}{)}
\end{Verbatim}


    \begin{Verbatim}[commandchars=\\\{\}]
<class 'pandas.core.frame.DataFrame'>
RangeIndex: 891 entries, 0 to 890
Data columns (total 12 columns):
PassengerId    891 non-null int64
Survived       891 non-null int64
Pclass         891 non-null int64
Name           891 non-null object
Sex            891 non-null object
Age            714 non-null float64
SibSp          891 non-null int64
Parch          891 non-null int64
Ticket         891 non-null object
Fare           891 non-null float64
Cabin          204 non-null object
Embarked       889 non-null object
dtypes: float64(2), int64(5), object(5)
memory usage: 83.6+ KB

    \end{Verbatim}

    Also a DataFrame has a \texttt{values} attribute, but attention: when
you have heterogeneous data, all values will be upcasted:

    \begin{Verbatim}[commandchars=\\\{\}]
{\color{incolor}In [{\color{incolor}18}]:} \PY{n}{df}\PY{o}{.}\PY{n}{values}
\end{Verbatim}


\begin{Verbatim}[commandchars=\\\{\}]
{\color{outcolor}Out[{\color{outcolor}18}]:} array([[1, 0, 3, {\ldots}, 7.25, nan, 'S'],
                [2, 1, 1, {\ldots}, 71.2833, 'C85', 'C'],
                [3, 1, 3, {\ldots}, 7.925, nan, 'S'],
                {\ldots},
                [889, 0, 3, {\ldots}, 23.45, nan, 'S'],
                [890, 1, 1, {\ldots}, 30.0, 'C148', 'C'],
                [891, 0, 3, {\ldots}, 7.75, nan, 'Q']], dtype=object)
\end{Verbatim}
            
    Apart from importing your data from an external source (text file,
excel, database, ..), one of the most common ways of creating a
dataframe is from a dictionary of arrays or lists.

Note that in the IPython notebook, the dataframe will display in a rich
HTML view:

    \begin{Verbatim}[commandchars=\\\{\}]
{\color{incolor}In [{\color{incolor}19}]:} \PY{n}{data} \PY{o}{=} \PY{p}{\PYZob{}}\PY{l+s+s1}{\PYZsq{}}\PY{l+s+s1}{country}\PY{l+s+s1}{\PYZsq{}}\PY{p}{:} \PY{p}{[}\PY{l+s+s1}{\PYZsq{}}\PY{l+s+s1}{Belgium}\PY{l+s+s1}{\PYZsq{}}\PY{p}{,} \PY{l+s+s1}{\PYZsq{}}\PY{l+s+s1}{France}\PY{l+s+s1}{\PYZsq{}}\PY{p}{,} \PY{l+s+s1}{\PYZsq{}}\PY{l+s+s1}{Germany}\PY{l+s+s1}{\PYZsq{}}\PY{p}{,} \PY{l+s+s1}{\PYZsq{}}\PY{l+s+s1}{Netherlands}\PY{l+s+s1}{\PYZsq{}}\PY{p}{,} \PY{l+s+s1}{\PYZsq{}}\PY{l+s+s1}{United Kingdom}\PY{l+s+s1}{\PYZsq{}}\PY{p}{]}\PY{p}{,}
                 \PY{l+s+s1}{\PYZsq{}}\PY{l+s+s1}{population}\PY{l+s+s1}{\PYZsq{}}\PY{p}{:} \PY{p}{[}\PY{l+m+mf}{11.3}\PY{p}{,} \PY{l+m+mf}{64.3}\PY{p}{,} \PY{l+m+mf}{81.3}\PY{p}{,} \PY{l+m+mf}{16.9}\PY{p}{,} \PY{l+m+mf}{64.9}\PY{p}{]}\PY{p}{,}
                 \PY{l+s+s1}{\PYZsq{}}\PY{l+s+s1}{area}\PY{l+s+s1}{\PYZsq{}}\PY{p}{:} \PY{p}{[}\PY{l+m+mi}{30510}\PY{p}{,} \PY{l+m+mi}{671308}\PY{p}{,} \PY{l+m+mi}{357050}\PY{p}{,} \PY{l+m+mi}{41526}\PY{p}{,} \PY{l+m+mi}{244820}\PY{p}{]}\PY{p}{,}
                 \PY{l+s+s1}{\PYZsq{}}\PY{l+s+s1}{capital}\PY{l+s+s1}{\PYZsq{}}\PY{p}{:} \PY{p}{[}\PY{l+s+s1}{\PYZsq{}}\PY{l+s+s1}{Brussels}\PY{l+s+s1}{\PYZsq{}}\PY{p}{,} \PY{l+s+s1}{\PYZsq{}}\PY{l+s+s1}{Paris}\PY{l+s+s1}{\PYZsq{}}\PY{p}{,} \PY{l+s+s1}{\PYZsq{}}\PY{l+s+s1}{Berlin}\PY{l+s+s1}{\PYZsq{}}\PY{p}{,} \PY{l+s+s1}{\PYZsq{}}\PY{l+s+s1}{Amsterdam}\PY{l+s+s1}{\PYZsq{}}\PY{p}{,} \PY{l+s+s1}{\PYZsq{}}\PY{l+s+s1}{London}\PY{l+s+s1}{\PYZsq{}}\PY{p}{]}\PY{p}{\PYZcb{}}
         \PY{n}{df\PYZus{}countries} \PY{o}{=} \PY{n}{pd}\PY{o}{.}\PY{n}{DataFrame}\PY{p}{(}\PY{n}{data}\PY{p}{)}
         \PY{n}{df\PYZus{}countries}
\end{Verbatim}


\begin{Verbatim}[commandchars=\\\{\}]
{\color{outcolor}Out[{\color{outcolor}19}]:}      area    capital         country  population
         0   30510   Brussels         Belgium        11.3
         1  671308      Paris          France        64.3
         2  357050     Berlin         Germany        81.3
         3   41526  Amsterdam     Netherlands        16.9
         4  244820     London  United Kingdom        64.9
\end{Verbatim}
            
    \hypertarget{one-dimensional-data-series-a-column-of-a-dataframe}{%
\subsubsection{\texorpdfstring{One-dimensional data: \texttt{Series} (a
column of a
DataFrame)}{One-dimensional data: Series (a column of a DataFrame)}}\label{one-dimensional-data-series-a-column-of-a-dataframe}}

A Series is a basic holder for \textbf{one-dimensional labeled data}.

    \begin{Verbatim}[commandchars=\\\{\}]
{\color{incolor}In [{\color{incolor}20}]:} \PY{n}{df}\PY{p}{[}\PY{l+s+s1}{\PYZsq{}}\PY{l+s+s1}{Age}\PY{l+s+s1}{\PYZsq{}}\PY{p}{]}
\end{Verbatim}


\begin{Verbatim}[commandchars=\\\{\}]
{\color{outcolor}Out[{\color{outcolor}20}]:} 0      22.0
         1      38.0
         2      26.0
         3      35.0
                {\ldots} 
         887    19.0
         888     NaN
         889    26.0
         890    32.0
         Name: Age, Length: 891, dtype: float64
\end{Verbatim}
            
    \begin{Verbatim}[commandchars=\\\{\}]
{\color{incolor}In [{\color{incolor}21}]:} \PY{n}{age} \PY{o}{=} \PY{n}{df}\PY{p}{[}\PY{l+s+s1}{\PYZsq{}}\PY{l+s+s1}{Age}\PY{l+s+s1}{\PYZsq{}}\PY{p}{]}
\end{Verbatim}


    \hypertarget{attributes-of-a-series-index-and-values}{%
\subsubsection{\texorpdfstring{Attributes of a Series: \texttt{index}
and
\texttt{values}}{Attributes of a Series: index and values}}\label{attributes-of-a-series-index-and-values}}

The Series has also an \texttt{index} and \texttt{values} attribute, but
no \texttt{columns}

    \begin{Verbatim}[commandchars=\\\{\}]
{\color{incolor}In [{\color{incolor}22}]:} \PY{n}{age}\PY{o}{.}\PY{n}{index}
\end{Verbatim}


\begin{Verbatim}[commandchars=\\\{\}]
{\color{outcolor}Out[{\color{outcolor}22}]:} RangeIndex(start=0, stop=891, step=1)
\end{Verbatim}
            
    You can access the underlying numpy array representation with the
\texttt{.values} attribute:

    \begin{Verbatim}[commandchars=\\\{\}]
{\color{incolor}In [{\color{incolor}23}]:} \PY{n}{age}\PY{o}{.}\PY{n}{values}\PY{p}{[}\PY{p}{:}\PY{l+m+mi}{10}\PY{p}{]}
\end{Verbatim}


\begin{Verbatim}[commandchars=\\\{\}]
{\color{outcolor}Out[{\color{outcolor}23}]:} array([22., 38., 26., 35., 35., nan, 54.,  2., 27., 14.])
\end{Verbatim}
            
    We can access series values via the index, just like for NumPy arrays:

    \begin{Verbatim}[commandchars=\\\{\}]
{\color{incolor}In [{\color{incolor}24}]:} \PY{n}{age}\PY{p}{[}\PY{l+m+mi}{0}\PY{p}{]}
\end{Verbatim}


\begin{Verbatim}[commandchars=\\\{\}]
{\color{outcolor}Out[{\color{outcolor}24}]:} 22.0
\end{Verbatim}
            
    Unlike the NumPy array, though, this index can be something other than
integers:

    \begin{Verbatim}[commandchars=\\\{\}]
{\color{incolor}In [{\color{incolor}25}]:} \PY{n}{df} \PY{o}{=} \PY{n}{df}\PY{o}{.}\PY{n}{set\PYZus{}index}\PY{p}{(}\PY{l+s+s1}{\PYZsq{}}\PY{l+s+s1}{Name}\PY{l+s+s1}{\PYZsq{}}\PY{p}{)}
         \PY{n}{df}
\end{Verbatim}


\begin{Verbatim}[commandchars=\\\{\}]
{\color{outcolor}Out[{\color{outcolor}25}]:}                                                     PassengerId  Survived  \textbackslash{}
         Name                                                                        
         Braund, Mr. Owen Harris                                       1         0   
         Cumings, Mrs. John Bradley (Florence Briggs Tha{\ldots}            2         1   
         Heikkinen, Miss. Laina                                        3         1   
         Futrelle, Mrs. Jacques Heath (Lily May Peel)                  4         1   
         {\ldots}                                                         {\ldots}       {\ldots}   
         Graham, Miss. Margaret Edith                                888         1   
         Johnston, Miss. Catherine Helen "Carrie"                    889         0   
         Behr, Mr. Karl Howell                                       890         1   
         Dooley, Mr. Patrick                                         891         0   
         
                                                             Pclass     Sex   Age  \textbackslash{}
         Name                                                                       
         Braund, Mr. Owen Harris                                  3    male  22.0   
         Cumings, Mrs. John Bradley (Florence Briggs Tha{\ldots}       1  female  38.0   
         Heikkinen, Miss. Laina                                   3  female  26.0   
         Futrelle, Mrs. Jacques Heath (Lily May Peel)             1  female  35.0   
         {\ldots}                                                    {\ldots}     {\ldots}   {\ldots}   
         Graham, Miss. Margaret Edith                             1  female  19.0   
         Johnston, Miss. Catherine Helen "Carrie"                 3  female   NaN   
         Behr, Mr. Karl Howell                                    1    male  26.0   
         Dooley, Mr. Patrick                                      3    male  32.0   
         
                                                             SibSp  Parch  \textbackslash{}
         Name                                                               
         Braund, Mr. Owen Harris                                 1      0   
         Cumings, Mrs. John Bradley (Florence Briggs Tha{\ldots}      1      0   
         Heikkinen, Miss. Laina                                  0      0   
         Futrelle, Mrs. Jacques Heath (Lily May Peel)            1      0   
         {\ldots}                                                   {\ldots}    {\ldots}   
         Graham, Miss. Margaret Edith                            0      0   
         Johnston, Miss. Catherine Helen "Carrie"                1      2   
         Behr, Mr. Karl Howell                                   0      0   
         Dooley, Mr. Patrick                                     0      0   
         
                                                                       Ticket     Fare  \textbackslash{}
         Name                                                                            
         Braund, Mr. Owen Harris                                    A/5 21171   7.2500   
         Cumings, Mrs. John Bradley (Florence Briggs Tha{\ldots}          PC 17599  71.2833   
         Heikkinen, Miss. Laina                              STON/O2. 3101282   7.9250   
         Futrelle, Mrs. Jacques Heath (Lily May Peel)                  113803  53.1000   
         {\ldots}                                                              {\ldots}      {\ldots}   
         Graham, Miss. Margaret Edith                                  112053  30.0000   
         Johnston, Miss. Catherine Helen "Carrie"                  W./C. 6607  23.4500   
         Behr, Mr. Karl Howell                                         111369  30.0000   
         Dooley, Mr. Patrick                                           370376   7.7500   
         
                                                            Cabin Embarked  
         Name                                                               
         Braund, Mr. Owen Harris                              NaN        S  
         Cumings, Mrs. John Bradley (Florence Briggs Tha{\ldots}   C85        C  
         Heikkinen, Miss. Laina                               NaN        S  
         Futrelle, Mrs. Jacques Heath (Lily May Peel)        C123        S  
         {\ldots}                                                  {\ldots}      {\ldots}  
         Graham, Miss. Margaret Edith                         B42        S  
         Johnston, Miss. Catherine Helen "Carrie"             NaN        S  
         Behr, Mr. Karl Howell                               C148        C  
         Dooley, Mr. Patrick                                  NaN        Q  
         
         [891 rows x 11 columns]
\end{Verbatim}
            
    \begin{Verbatim}[commandchars=\\\{\}]
{\color{incolor}In [{\color{incolor}26}]:} \PY{n}{age} \PY{o}{=} \PY{n}{df}\PY{p}{[}\PY{l+s+s1}{\PYZsq{}}\PY{l+s+s1}{Age}\PY{l+s+s1}{\PYZsq{}}\PY{p}{]}
         \PY{n}{age}
\end{Verbatim}


\begin{Verbatim}[commandchars=\\\{\}]
{\color{outcolor}Out[{\color{outcolor}26}]:} Name
         Braund, Mr. Owen Harris                                22.0
         Cumings, Mrs. John Bradley (Florence Briggs Thayer)    38.0
         Heikkinen, Miss. Laina                                 26.0
         Futrelle, Mrs. Jacques Heath (Lily May Peel)           35.0
                                                                {\ldots} 
         Graham, Miss. Margaret Edith                           19.0
         Johnston, Miss. Catherine Helen "Carrie"                NaN
         Behr, Mr. Karl Howell                                  26.0
         Dooley, Mr. Patrick                                    32.0
         Name: Age, Length: 891, dtype: float64
\end{Verbatim}
            
    \begin{Verbatim}[commandchars=\\\{\}]
{\color{incolor}In [{\color{incolor}27}]:} \PY{n}{age}\PY{p}{[}\PY{l+s+s1}{\PYZsq{}}\PY{l+s+s1}{Dooley, Mr. Patrick}\PY{l+s+s1}{\PYZsq{}}\PY{p}{]}
\end{Verbatim}


\begin{Verbatim}[commandchars=\\\{\}]
{\color{outcolor}Out[{\color{outcolor}27}]:} 32.0
\end{Verbatim}
            
    but with the power of numpy arrays. Many things you can do with numpy
arrays, can also be applied on DataFrames / Series.

Eg element-wise operations:

    \begin{Verbatim}[commandchars=\\\{\}]
{\color{incolor}In [{\color{incolor}28}]:} \PY{n}{age} \PY{o}{*} \PY{l+m+mi}{1000}
\end{Verbatim}


\begin{Verbatim}[commandchars=\\\{\}]
{\color{outcolor}Out[{\color{outcolor}28}]:} Name
         Braund, Mr. Owen Harris                                22000.0
         Cumings, Mrs. John Bradley (Florence Briggs Thayer)    38000.0
         Heikkinen, Miss. Laina                                 26000.0
         Futrelle, Mrs. Jacques Heath (Lily May Peel)           35000.0
                                                                 {\ldots}   
         Graham, Miss. Margaret Edith                           19000.0
         Johnston, Miss. Catherine Helen "Carrie"                   NaN
         Behr, Mr. Karl Howell                                  26000.0
         Dooley, Mr. Patrick                                    32000.0
         Name: Age, Length: 891, dtype: float64
\end{Verbatim}
            
    A range of methods:

    \begin{Verbatim}[commandchars=\\\{\}]
{\color{incolor}In [{\color{incolor}29}]:} \PY{n}{age}\PY{o}{.}\PY{n}{mean}\PY{p}{(}\PY{p}{)}
\end{Verbatim}


\begin{Verbatim}[commandchars=\\\{\}]
{\color{outcolor}Out[{\color{outcolor}29}]:} 29.69911764705882
\end{Verbatim}
            
    Fancy indexing, like indexing with a list or boolean indexing:

    \begin{Verbatim}[commandchars=\\\{\}]
{\color{incolor}In [{\color{incolor}30}]:} \PY{n}{age}\PY{p}{[}\PY{n}{age} \PY{o}{\PYZgt{}} \PY{l+m+mi}{70}\PY{p}{]}
\end{Verbatim}


\begin{Verbatim}[commandchars=\\\{\}]
{\color{outcolor}Out[{\color{outcolor}30}]:} Name
         Goldschmidt, Mr. George B               71.0
         Connors, Mr. Patrick                    70.5
         Artagaveytia, Mr. Ramon                 71.0
         Barkworth, Mr. Algernon Henry Wilson    80.0
         Svensson, Mr. Johan                     74.0
         Name: Age, dtype: float64
\end{Verbatim}
            
    But also a lot of pandas specific methods, e.g.

    \begin{Verbatim}[commandchars=\\\{\}]
{\color{incolor}In [{\color{incolor}31}]:} \PY{n}{df}\PY{p}{[}\PY{l+s+s1}{\PYZsq{}}\PY{l+s+s1}{Embarked}\PY{l+s+s1}{\PYZsq{}}\PY{p}{]}\PY{o}{.}\PY{n}{value\PYZus{}counts}\PY{p}{(}\PY{p}{)}
\end{Verbatim}


\begin{Verbatim}[commandchars=\\\{\}]
{\color{outcolor}Out[{\color{outcolor}31}]:} S    644
         C    168
         Q     77
         Name: Embarked, dtype: int64
\end{Verbatim}
            
    \hypertarget{data-import-and-export}{%
\section{Data import and export}\label{data-import-and-export}}

    A wide range of input/output formats are natively supported by pandas:

\begin{itemize}
\tightlist
\item
  CSV, text
\item
  SQL database
\item
  Excel
\item
  HDF5
\item
  json
\item
  html
\item
  pickle
\item
  sas, stata
\item
  (parquet)
\item
  \ldots{}
\end{itemize}

    Very powerful csv reader:

    \begin{Verbatim}[commandchars=\\\{\}]
{\color{incolor}In [{\color{incolor}33}]:} pd.read\PYZus{}csv\PY{o}{?}
\end{Verbatim}


    Luckily, if we have a well formed csv file, we don't need many of those
arguments:

    \begin{Verbatim}[commandchars=\\\{\}]
{\color{incolor}In [{\color{incolor}34}]:} \PY{n}{df} \PY{o}{=} \PY{n}{pd}\PY{o}{.}\PY{n}{read\PYZus{}csv}\PY{p}{(}\PY{l+s+s2}{\PYZdq{}}\PY{l+s+s2}{titanic\PYZhy{}train.csv}\PY{l+s+s2}{\PYZdq{}}\PY{p}{)}
\end{Verbatim}


    \begin{Verbatim}[commandchars=\\\{\}]
{\color{incolor}In [{\color{incolor}35}]:} \PY{n}{df}\PY{o}{.}\PY{n}{head}\PY{p}{(}\PY{p}{)}
\end{Verbatim}


\begin{Verbatim}[commandchars=\\\{\}]
{\color{outcolor}Out[{\color{outcolor}35}]:}    PassengerId  Survived  Pclass  \textbackslash{}
         0            1         0       3   
         1            2         1       1   
         2            3         1       3   
         3            4         1       1   
         4            5         0       3   
         
                                                         Name     Sex   Age  SibSp  \textbackslash{}
         0                            Braund, Mr. Owen Harris    male  22.0      1   
         1  Cumings, Mrs. John Bradley (Florence Briggs Th{\ldots}  female  38.0      1   
         2                             Heikkinen, Miss. Laina  female  26.0      0   
         3       Futrelle, Mrs. Jacques Heath (Lily May Peel)  female  35.0      1   
         4                           Allen, Mr. William Henry    male  35.0      0   
         
            Parch            Ticket     Fare Cabin Embarked  
         0      0         A/5 21171   7.2500   NaN        S  
         1      0          PC 17599  71.2833   C85        C  
         2      0  STON/O2. 3101282   7.9250   NaN        S  
         3      0            113803  53.1000  C123        S  
         4      0            373450   8.0500   NaN        S  
\end{Verbatim}
            
    The default plot (when not specifying \texttt{kind}) is a line plot of
all columns:

    \hypertarget{selecting-and-filtering-data}{%
\section{Selecting and filtering
data}\label{selecting-and-filtering-data}}

    ATTENTION!:

One of pandas' basic features is the labeling of rows and columns, but
this makes indexing also a bit more complex compared to numpy. We now
have to distuinguish between:

selection by \textbf{label}

selection by \textbf{position}

    \hypertarget{df-provides-some-convenience-shortcuts}{%
\subsubsection{\texorpdfstring{\texttt{df{[}{]}} provides some
convenience
shortcuts}{df{[}{]} provides some convenience shortcuts}}\label{df-provides-some-convenience-shortcuts}}

    For a DataFrame, basic indexing selects the columns.

Selecting a single column:

    \begin{Verbatim}[commandchars=\\\{\}]
{\color{incolor}In [{\color{incolor}36}]:} \PY{n}{df}\PY{p}{[}\PY{l+s+s1}{\PYZsq{}}\PY{l+s+s1}{Age}\PY{l+s+s1}{\PYZsq{}}\PY{p}{]}
\end{Verbatim}


\begin{Verbatim}[commandchars=\\\{\}]
{\color{outcolor}Out[{\color{outcolor}36}]:} 0      22.0
         1      38.0
         2      26.0
         3      35.0
                {\ldots} 
         887    19.0
         888     NaN
         889    26.0
         890    32.0
         Name: Age, Length: 891, dtype: float64
\end{Verbatim}
            
    or multiple columns:

    \begin{Verbatim}[commandchars=\\\{\}]
{\color{incolor}In [{\color{incolor}37}]:} \PY{n}{df}\PY{p}{[}\PY{p}{[}\PY{l+s+s1}{\PYZsq{}}\PY{l+s+s1}{Age}\PY{l+s+s1}{\PYZsq{}}\PY{p}{,} \PY{l+s+s1}{\PYZsq{}}\PY{l+s+s1}{Fare}\PY{l+s+s1}{\PYZsq{}}\PY{p}{]}\PY{p}{]}
\end{Verbatim}


\begin{Verbatim}[commandchars=\\\{\}]
{\color{outcolor}Out[{\color{outcolor}37}]:}       Age     Fare
         0    22.0   7.2500
         1    38.0  71.2833
         2    26.0   7.9250
         3    35.0  53.1000
         ..    {\ldots}      {\ldots}
         887  19.0  30.0000
         888   NaN  23.4500
         889  26.0  30.0000
         890  32.0   7.7500
         
         [891 rows x 2 columns]
\end{Verbatim}
            
    But, slicing accesses the rows:

    \begin{Verbatim}[commandchars=\\\{\}]
{\color{incolor}In [{\color{incolor}38}]:} \PY{n}{df}\PY{p}{[}\PY{l+m+mi}{10}\PY{p}{:}\PY{l+m+mi}{15}\PY{p}{]}
\end{Verbatim}


\begin{Verbatim}[commandchars=\\\{\}]
{\color{outcolor}Out[{\color{outcolor}38}]:}     PassengerId  Survived  Pclass                                  Name  \textbackslash{}
         10           11         1       3       Sandstrom, Miss. Marguerite Rut   
         11           12         1       1              Bonnell, Miss. Elizabeth   
         12           13         0       3        Saundercock, Mr. William Henry   
         13           14         0       3           Andersson, Mr. Anders Johan   
         14           15         0       3  Vestrom, Miss. Hulda Amanda Adolfina   
         
                Sex   Age  SibSp  Parch     Ticket     Fare Cabin Embarked  
         10  female   4.0      1      1    PP 9549  16.7000    G6        S  
         11  female  58.0      0      0     113783  26.5500  C103        S  
         12    male  20.0      0      0  A/5. 2151   8.0500   NaN        S  
         13    male  39.0      1      5     347082  31.2750   NaN        S  
         14  female  14.0      0      0     350406   7.8542   NaN        S  
\end{Verbatim}
            
    \hypertarget{systematic-indexing-with-loc-and-iloc}{%
\subsubsection{\texorpdfstring{Systematic indexing with \texttt{loc} and
\texttt{iloc}}{Systematic indexing with loc and iloc}}\label{systematic-indexing-with-loc-and-iloc}}

When using \texttt{{[}{]}} like above, you can only select from one axis
at once (rows or columns, not both). For more advanced indexing, you
have some extra attributes:

\begin{itemize}
\tightlist
\item
  \texttt{loc}: selection by label
\item
  \texttt{iloc}: selection by position
\end{itemize}

    \begin{Verbatim}[commandchars=\\\{\}]
{\color{incolor}In [{\color{incolor}39}]:} \PY{n}{df} \PY{o}{=} \PY{n}{df}\PY{o}{.}\PY{n}{set\PYZus{}index}\PY{p}{(}\PY{l+s+s1}{\PYZsq{}}\PY{l+s+s1}{Name}\PY{l+s+s1}{\PYZsq{}}\PY{p}{)}
\end{Verbatim}


    \begin{Verbatim}[commandchars=\\\{\}]
{\color{incolor}In [{\color{incolor}40}]:} \PY{n}{df}\PY{o}{.}\PY{n}{loc}\PY{p}{[}\PY{l+s+s1}{\PYZsq{}}\PY{l+s+s1}{Bonnell, Miss. Elizabeth}\PY{l+s+s1}{\PYZsq{}}\PY{p}{,} \PY{l+s+s1}{\PYZsq{}}\PY{l+s+s1}{Fare}\PY{l+s+s1}{\PYZsq{}}\PY{p}{]}
\end{Verbatim}


\begin{Verbatim}[commandchars=\\\{\}]
{\color{outcolor}Out[{\color{outcolor}40}]:} 26.55
\end{Verbatim}
            
    \begin{Verbatim}[commandchars=\\\{\}]
{\color{incolor}In [{\color{incolor}41}]:} \PY{n}{df}\PY{o}{.}\PY{n}{loc}\PY{p}{[}\PY{l+s+s1}{\PYZsq{}}\PY{l+s+s1}{Bonnell, Miss. Elizabeth}\PY{l+s+s1}{\PYZsq{}}\PY{p}{:}\PY{l+s+s1}{\PYZsq{}}\PY{l+s+s1}{Andersson, Mr. Anders Johan}\PY{l+s+s1}{\PYZsq{}}\PY{p}{,} \PY{p}{:}\PY{p}{]}
\end{Verbatim}


\begin{Verbatim}[commandchars=\\\{\}]
{\color{outcolor}Out[{\color{outcolor}41}]:}                                 PassengerId  Survived  Pclass     Sex   Age  \textbackslash{}
         Name                                                                          
         Bonnell, Miss. Elizabeth                 12         1       1  female  58.0   
         Saundercock, Mr. William Henry           13         0       3    male  20.0   
         Andersson, Mr. Anders Johan              14         0       3    male  39.0   
         
                                         SibSp  Parch     Ticket    Fare Cabin Embarked  
         Name                                                                            
         Bonnell, Miss. Elizabeth            0      0     113783  26.550  C103        S  
         Saundercock, Mr. William Henry      0      0  A/5. 2151   8.050   NaN        S  
         Andersson, Mr. Anders Johan         1      5     347082  31.275   NaN        S  
\end{Verbatim}
            
    Selecting by position with \texttt{iloc} works similar as indexing numpy
arrays:

    \begin{Verbatim}[commandchars=\\\{\}]
{\color{incolor}In [{\color{incolor}42}]:} \PY{n}{df}\PY{o}{.}\PY{n}{iloc}\PY{p}{[}\PY{l+m+mi}{0}\PY{p}{:}\PY{l+m+mi}{2}\PY{p}{,}\PY{l+m+mi}{1}\PY{p}{:}\PY{l+m+mi}{3}\PY{p}{]}
\end{Verbatim}


\begin{Verbatim}[commandchars=\\\{\}]
{\color{outcolor}Out[{\color{outcolor}42}]:}                                                     Survived  Pclass
         Name                                                                
         Braund, Mr. Owen Harris                                    0       3
         Cumings, Mrs. John Bradley (Florence Briggs Tha{\ldots}         1       1
\end{Verbatim}
            
    The different indexing methods can also be used to assign data:

    \begin{Verbatim}[commandchars=\\\{\}]
{\color{incolor}In [{\color{incolor}43}]:} \PY{n}{df}\PY{o}{.}\PY{n}{loc}\PY{p}{[}\PY{l+s+s1}{\PYZsq{}}\PY{l+s+s1}{Braund, Mr. Owen Harris}\PY{l+s+s1}{\PYZsq{}}\PY{p}{,} \PY{l+s+s1}{\PYZsq{}}\PY{l+s+s1}{Survived}\PY{l+s+s1}{\PYZsq{}}\PY{p}{]} \PY{o}{=} \PY{l+m+mi}{100}
\end{Verbatim}


    \begin{Verbatim}[commandchars=\\\{\}]
{\color{incolor}In [{\color{incolor}44}]:} \PY{n}{df}
\end{Verbatim}


\begin{Verbatim}[commandchars=\\\{\}]
{\color{outcolor}Out[{\color{outcolor}44}]:}                                                     PassengerId  Survived  \textbackslash{}
         Name                                                                        
         Braund, Mr. Owen Harris                                       1       100   
         Cumings, Mrs. John Bradley (Florence Briggs Tha{\ldots}            2         1   
         Heikkinen, Miss. Laina                                        3         1   
         Futrelle, Mrs. Jacques Heath (Lily May Peel)                  4         1   
         {\ldots}                                                         {\ldots}       {\ldots}   
         Graham, Miss. Margaret Edith                                888         1   
         Johnston, Miss. Catherine Helen "Carrie"                    889         0   
         Behr, Mr. Karl Howell                                       890         1   
         Dooley, Mr. Patrick                                         891         0   
         
                                                             Pclass     Sex   Age  \textbackslash{}
         Name                                                                       
         Braund, Mr. Owen Harris                                  3    male  22.0   
         Cumings, Mrs. John Bradley (Florence Briggs Tha{\ldots}       1  female  38.0   
         Heikkinen, Miss. Laina                                   3  female  26.0   
         Futrelle, Mrs. Jacques Heath (Lily May Peel)             1  female  35.0   
         {\ldots}                                                    {\ldots}     {\ldots}   {\ldots}   
         Graham, Miss. Margaret Edith                             1  female  19.0   
         Johnston, Miss. Catherine Helen "Carrie"                 3  female   NaN   
         Behr, Mr. Karl Howell                                    1    male  26.0   
         Dooley, Mr. Patrick                                      3    male  32.0   
         
                                                             SibSp  Parch  \textbackslash{}
         Name                                                               
         Braund, Mr. Owen Harris                                 1      0   
         Cumings, Mrs. John Bradley (Florence Briggs Tha{\ldots}      1      0   
         Heikkinen, Miss. Laina                                  0      0   
         Futrelle, Mrs. Jacques Heath (Lily May Peel)            1      0   
         {\ldots}                                                   {\ldots}    {\ldots}   
         Graham, Miss. Margaret Edith                            0      0   
         Johnston, Miss. Catherine Helen "Carrie"                1      2   
         Behr, Mr. Karl Howell                                   0      0   
         Dooley, Mr. Patrick                                     0      0   
         
                                                                       Ticket     Fare  \textbackslash{}
         Name                                                                            
         Braund, Mr. Owen Harris                                    A/5 21171   7.2500   
         Cumings, Mrs. John Bradley (Florence Briggs Tha{\ldots}          PC 17599  71.2833   
         Heikkinen, Miss. Laina                              STON/O2. 3101282   7.9250   
         Futrelle, Mrs. Jacques Heath (Lily May Peel)                  113803  53.1000   
         {\ldots}                                                              {\ldots}      {\ldots}   
         Graham, Miss. Margaret Edith                                  112053  30.0000   
         Johnston, Miss. Catherine Helen "Carrie"                  W./C. 6607  23.4500   
         Behr, Mr. Karl Howell                                         111369  30.0000   
         Dooley, Mr. Patrick                                           370376   7.7500   
         
                                                            Cabin Embarked  
         Name                                                               
         Braund, Mr. Owen Harris                              NaN        S  
         Cumings, Mrs. John Bradley (Florence Briggs Tha{\ldots}   C85        C  
         Heikkinen, Miss. Laina                               NaN        S  
         Futrelle, Mrs. Jacques Heath (Lily May Peel)        C123        S  
         {\ldots}                                                  {\ldots}      {\ldots}  
         Graham, Miss. Margaret Edith                         B42        S  
         Johnston, Miss. Catherine Helen "Carrie"             NaN        S  
         Behr, Mr. Karl Howell                               C148        C  
         Dooley, Mr. Patrick                                  NaN        Q  
         
         [891 rows x 11 columns]
\end{Verbatim}
            
    \hypertarget{boolean-indexing-filtering}{%
\subsubsection{Boolean indexing
(filtering)}\label{boolean-indexing-filtering}}

    Often, you want to select rows based on a certain condition. This can be
done with `boolean indexing' (like a where clause in SQL) and comparable
to numpy.

The indexer (or boolean mask) should be 1-dimensional and the same
length as the thing being indexed.

    \begin{Verbatim}[commandchars=\\\{\}]
{\color{incolor}In [{\color{incolor}45}]:} \PY{n}{df}\PY{p}{[}\PY{l+s+s1}{\PYZsq{}}\PY{l+s+s1}{Fare}\PY{l+s+s1}{\PYZsq{}}\PY{p}{]} \PY{o}{\PYZgt{}} \PY{l+m+mi}{50}
\end{Verbatim}


\begin{Verbatim}[commandchars=\\\{\}]
{\color{outcolor}Out[{\color{outcolor}45}]:} Name
         Braund, Mr. Owen Harris                                False
         Cumings, Mrs. John Bradley (Florence Briggs Thayer)     True
         Heikkinen, Miss. Laina                                 False
         Futrelle, Mrs. Jacques Heath (Lily May Peel)            True
                                                                {\ldots}  
         Graham, Miss. Margaret Edith                           False
         Johnston, Miss. Catherine Helen "Carrie"               False
         Behr, Mr. Karl Howell                                  False
         Dooley, Mr. Patrick                                    False
         Name: Fare, Length: 891, dtype: bool
\end{Verbatim}
            
    \begin{Verbatim}[commandchars=\\\{\}]
{\color{incolor}In [{\color{incolor}46}]:} \PY{n}{df}\PY{p}{[}\PY{n}{df}\PY{p}{[}\PY{l+s+s1}{\PYZsq{}}\PY{l+s+s1}{Fare}\PY{l+s+s1}{\PYZsq{}}\PY{p}{]} \PY{o}{\PYZgt{}} \PY{l+m+mi}{50}\PY{p}{]}
\end{Verbatim}


\begin{Verbatim}[commandchars=\\\{\}]
{\color{outcolor}Out[{\color{outcolor}46}]:}                                                     PassengerId  Survived  \textbackslash{}
         Name                                                                        
         Cumings, Mrs. John Bradley (Florence Briggs Tha{\ldots}            2         1   
         Futrelle, Mrs. Jacques Heath (Lily May Peel)                  4         1   
         McCarthy, Mr. Timothy J                                       7         0   
         Fortune, Mr. Charles Alexander                               28         0   
         {\ldots}                                                         {\ldots}       {\ldots}   
         Sage, Miss. Dorothy Edith "Dolly"                           864         0   
         Roebling, Mr. Washington Augustus II                        868         0   
         Beckwith, Mrs. Richard Leonard (Sallie Monypeny)            872         1   
         Potter, Mrs. Thomas Jr (Lily Alexenia Wilson)               880         1   
         
                                                             Pclass     Sex   Age  \textbackslash{}
         Name                                                                       
         Cumings, Mrs. John Bradley (Florence Briggs Tha{\ldots}       1  female  38.0   
         Futrelle, Mrs. Jacques Heath (Lily May Peel)             1  female  35.0   
         McCarthy, Mr. Timothy J                                  1    male  54.0   
         Fortune, Mr. Charles Alexander                           1    male  19.0   
         {\ldots}                                                    {\ldots}     {\ldots}   {\ldots}   
         Sage, Miss. Dorothy Edith "Dolly"                        3  female   NaN   
         Roebling, Mr. Washington Augustus II                     1    male  31.0   
         Beckwith, Mrs. Richard Leonard (Sallie Monypeny)         1  female  47.0   
         Potter, Mrs. Thomas Jr (Lily Alexenia Wilson)            1  female  56.0   
         
                                                             SibSp  Parch    Ticket  \textbackslash{}
         Name                                                                         
         Cumings, Mrs. John Bradley (Florence Briggs Tha{\ldots}      1      0  PC 17599   
         Futrelle, Mrs. Jacques Heath (Lily May Peel)            1      0    113803   
         McCarthy, Mr. Timothy J                                 0      0     17463   
         Fortune, Mr. Charles Alexander                          3      2     19950   
         {\ldots}                                                   {\ldots}    {\ldots}       {\ldots}   
         Sage, Miss. Dorothy Edith "Dolly"                       8      2  CA. 2343   
         Roebling, Mr. Washington Augustus II                    0      0  PC 17590   
         Beckwith, Mrs. Richard Leonard (Sallie Monypeny)        1      1     11751   
         Potter, Mrs. Thomas Jr (Lily Alexenia Wilson)           0      1     11767   
         
                                                                 Fare        Cabin  \textbackslash{}
         Name                                                                        
         Cumings, Mrs. John Bradley (Florence Briggs Tha{\ldots}   71.2833          C85   
         Futrelle, Mrs. Jacques Heath (Lily May Peel)         53.1000         C123   
         McCarthy, Mr. Timothy J                              51.8625          E46   
         Fortune, Mr. Charles Alexander                      263.0000  C23 C25 C27   
         {\ldots}                                                      {\ldots}          {\ldots}   
         Sage, Miss. Dorothy Edith "Dolly"                    69.5500          NaN   
         Roebling, Mr. Washington Augustus II                 50.4958          A24   
         Beckwith, Mrs. Richard Leonard (Sallie Monypeny)     52.5542          D35   
         Potter, Mrs. Thomas Jr (Lily Alexenia Wilson)        83.1583          C50   
         
                                                            Embarked  
         Name                                                         
         Cumings, Mrs. John Bradley (Florence Briggs Tha{\ldots}        C  
         Futrelle, Mrs. Jacques Heath (Lily May Peel)              S  
         McCarthy, Mr. Timothy J                                   S  
         Fortune, Mr. Charles Alexander                            S  
         {\ldots}                                                     {\ldots}  
         Sage, Miss. Dorothy Edith "Dolly"                         S  
         Roebling, Mr. Washington Augustus II                      S  
         Beckwith, Mrs. Richard Leonard (Sallie Monypeny)          S  
         Potter, Mrs. Thomas Jr (Lily Alexenia Wilson)             C  
         
         [160 rows x 11 columns]
\end{Verbatim}
            
    \begin{Verbatim}[commandchars=\\\{\}]
{\color{incolor}In [{\color{incolor}47}]:} \PY{n}{df} \PY{o}{=} \PY{n}{pd}\PY{o}{.}\PY{n}{read\PYZus{}csv}\PY{p}{(}\PY{l+s+s2}{\PYZdq{}}\PY{l+s+s2}{titanic\PYZhy{}train.csv}\PY{l+s+s2}{\PYZdq{}}\PY{p}{)}
\end{Verbatim}


    \hypertarget{the-group-by-operation}{%
\section{The group-by operation}\label{the-group-by-operation}}

    \hypertarget{some-theory-the-groupby-operation-split-apply-combine}{%
\subsubsection{Some `theory': the groupby operation
(split-apply-combine)}\label{some-theory-the-groupby-operation-split-apply-combine}}

    \begin{Verbatim}[commandchars=\\\{\}]
{\color{incolor}In [{\color{incolor}49}]:} \PY{n}{df} \PY{o}{=} \PY{n}{pd}\PY{o}{.}\PY{n}{DataFrame}\PY{p}{(}\PY{p}{\PYZob{}}\PY{l+s+s1}{\PYZsq{}}\PY{l+s+s1}{key}\PY{l+s+s1}{\PYZsq{}}\PY{p}{:}\PY{p}{[}\PY{l+s+s1}{\PYZsq{}}\PY{l+s+s1}{A}\PY{l+s+s1}{\PYZsq{}}\PY{p}{,}\PY{l+s+s1}{\PYZsq{}}\PY{l+s+s1}{B}\PY{l+s+s1}{\PYZsq{}}\PY{p}{,}\PY{l+s+s1}{\PYZsq{}}\PY{l+s+s1}{C}\PY{l+s+s1}{\PYZsq{}}\PY{p}{,}\PY{l+s+s1}{\PYZsq{}}\PY{l+s+s1}{A}\PY{l+s+s1}{\PYZsq{}}\PY{p}{,}\PY{l+s+s1}{\PYZsq{}}\PY{l+s+s1}{B}\PY{l+s+s1}{\PYZsq{}}\PY{p}{,}\PY{l+s+s1}{\PYZsq{}}\PY{l+s+s1}{C}\PY{l+s+s1}{\PYZsq{}}\PY{p}{,}\PY{l+s+s1}{\PYZsq{}}\PY{l+s+s1}{A}\PY{l+s+s1}{\PYZsq{}}\PY{p}{,}\PY{l+s+s1}{\PYZsq{}}\PY{l+s+s1}{B}\PY{l+s+s1}{\PYZsq{}}\PY{p}{,}\PY{l+s+s1}{\PYZsq{}}\PY{l+s+s1}{C}\PY{l+s+s1}{\PYZsq{}}\PY{p}{]}\PY{p}{,}
                            \PY{l+s+s1}{\PYZsq{}}\PY{l+s+s1}{data}\PY{l+s+s1}{\PYZsq{}}\PY{p}{:} \PY{p}{[}\PY{l+m+mi}{0}\PY{p}{,} \PY{l+m+mi}{5}\PY{p}{,} \PY{l+m+mi}{10}\PY{p}{,} \PY{l+m+mi}{5}\PY{p}{,} \PY{l+m+mi}{10}\PY{p}{,} \PY{l+m+mi}{15}\PY{p}{,} \PY{l+m+mi}{10}\PY{p}{,} \PY{l+m+mi}{15}\PY{p}{,} \PY{l+m+mi}{20}\PY{p}{]}\PY{p}{\PYZcb{}}\PY{p}{)}
         \PY{n}{df}
\end{Verbatim}


\begin{Verbatim}[commandchars=\\\{\}]
{\color{outcolor}Out[{\color{outcolor}49}]:}     data key
         0      0   A
         1      5   B
         2     10   C
         3      5   A
         ..   {\ldots}  ..
         5     15   C
         6     10   A
         7     15   B
         8     20   C
         
         [9 rows x 2 columns]
\end{Verbatim}
            
    \hypertarget{recap-aggregating-functions}{%
\subsubsection{Recap: aggregating
functions}\label{recap-aggregating-functions}}

    When analyzing data, you often calculate summary statistics
(aggregations like the mean, max, \ldots{}). As we have seen before, we
can easily calculate such a statistic for a Series or column using one
of the many available methods. For example:

    \begin{Verbatim}[commandchars=\\\{\}]
{\color{incolor}In [{\color{incolor}50}]:} \PY{n}{df}\PY{p}{[}\PY{l+s+s1}{\PYZsq{}}\PY{l+s+s1}{data}\PY{l+s+s1}{\PYZsq{}}\PY{p}{]}\PY{o}{.}\PY{n}{sum}\PY{p}{(}\PY{p}{)}
\end{Verbatim}


\begin{Verbatim}[commandchars=\\\{\}]
{\color{outcolor}Out[{\color{outcolor}50}]:} 90
\end{Verbatim}
            
    However, in many cases your data has certain groups in it, and in that
case, you may want to calculate this statistic for each of the groups.

For example, in the above dataframe \texttt{df}, there is a column `key'
which has three possible values: `A', `B' and `C'. When we want to
calculate the sum for each of those groups, we could do the following:

    \begin{Verbatim}[commandchars=\\\{\}]
{\color{incolor}In [{\color{incolor}55}]:} \PY{k}{for} \PY{n}{key} \PY{o+ow}{in} \PY{p}{[}\PY{l+s+s1}{\PYZsq{}}\PY{l+s+s1}{A}\PY{l+s+s1}{\PYZsq{}}\PY{p}{,} \PY{l+s+s1}{\PYZsq{}}\PY{l+s+s1}{B}\PY{l+s+s1}{\PYZsq{}}\PY{p}{,} \PY{l+s+s1}{\PYZsq{}}\PY{l+s+s1}{C}\PY{l+s+s1}{\PYZsq{}}\PY{p}{]}\PY{p}{:}
             \PY{n+nb}{print}\PY{p}{(}\PY{n}{key}\PY{p}{,} \PY{n}{df}\PY{p}{[}\PY{n}{df}\PY{p}{[}\PY{l+s+s1}{\PYZsq{}}\PY{l+s+s1}{key}\PY{l+s+s1}{\PYZsq{}}\PY{p}{]} \PY{o}{==} \PY{n}{key}\PY{p}{]}\PY{p}{[}\PY{l+s+s1}{\PYZsq{}}\PY{l+s+s1}{data}\PY{l+s+s1}{\PYZsq{}}\PY{p}{]}\PY{o}{.}\PY{n}{sum}\PY{p}{(}\PY{p}{)}\PY{p}{)}
             
\end{Verbatim}


    \begin{Verbatim}[commandchars=\\\{\}]
A 15
B 30
C 45

    \end{Verbatim}

    This becomes very verbose when having multiple groups. You could make
the above a bit easier by looping over the different values, but still,
it is not very convenient to work with.

What we did above, applying a function on different groups, is a
``groupby operation'', and pandas provides some convenient functionality
for this.

    \hypertarget{groupby-applying-functions-per-group}{%
\subsubsection{Groupby: applying functions per
group}\label{groupby-applying-functions-per-group}}

    The ``group by'' concept: we want to \textbf{apply the same function on
subsets of your dataframe, based on some key to split the dataframe in
subsets}

This operation is also referred to as the ``split-apply-combine''
operation, involving the following steps:

\begin{itemize}
\tightlist
\item
  \textbf{Splitting} the data into groups based on some criteria
\item
  \textbf{Applying} a function to each group independently
\item
  \textbf{Combining} the results into a data structure
\end{itemize}

Similar to SQL \texttt{GROUP\ BY}

    Instead of doing the manual filtering as above

\begin{verbatim}
df[df['key'] == "A"].sum()
df[df['key'] == "B"].sum()
...
\end{verbatim}

pandas provides the \texttt{groupby} method to do exactly this:

    \begin{Verbatim}[commandchars=\\\{\}]
{\color{incolor}In [{\color{incolor}59}]:} \PY{n}{df}\PY{o}{.}\PY{n}{groupby}\PY{p}{(}\PY{l+s+s1}{\PYZsq{}}\PY{l+s+s1}{key}\PY{l+s+s1}{\PYZsq{}}\PY{p}{)}\PY{o}{.}\PY{n}{sum}\PY{p}{(}\PY{p}{)}\PY{o}{.}\PY{n}{plot}\PY{p}{(}\PY{n}{kind}\PY{o}{=}\PY{l+s+s1}{\PYZsq{}}\PY{l+s+s1}{bar}\PY{l+s+s1}{\PYZsq{}}\PY{p}{)}
\end{Verbatim}


\begin{Verbatim}[commandchars=\\\{\}]
{\color{outcolor}Out[{\color{outcolor}59}]:} <matplotlib.axes.\_subplots.AxesSubplot at 0x10e56c4e0>
\end{Verbatim}
            
    \begin{center}
    \adjustimage{max size={0.9\linewidth}{0.9\paperheight}}{output_91_1.png}
    \end{center}
    { \hspace*{\fill} \\}
    
    \begin{Verbatim}[commandchars=\\\{\}]
{\color{incolor}In [{\color{incolor}60}]:} \PY{n}{df}\PY{o}{.}\PY{n}{groupby}\PY{p}{(}\PY{l+s+s1}{\PYZsq{}}\PY{l+s+s1}{key}\PY{l+s+s1}{\PYZsq{}}\PY{p}{)}\PY{o}{.}\PY{n}{aggregate}\PY{p}{(}\PY{n}{np}\PY{o}{.}\PY{n}{sum}\PY{p}{)}  \PY{c+c1}{\PYZsh{} \PYZsq{}sum\PYZsq{}}
\end{Verbatim}


\begin{Verbatim}[commandchars=\\\{\}]
{\color{outcolor}Out[{\color{outcolor}60}]:}      data
         key      
         A      15
         B      30
         C      45
\end{Verbatim}
            
    And many more methods are available.

    \begin{Verbatim}[commandchars=\\\{\}]
{\color{incolor}In [{\color{incolor}61}]:} \PY{n}{df}\PY{o}{.}\PY{n}{groupby}\PY{p}{(}\PY{l+s+s1}{\PYZsq{}}\PY{l+s+s1}{key}\PY{l+s+s1}{\PYZsq{}}\PY{p}{)}\PY{p}{[}\PY{l+s+s1}{\PYZsq{}}\PY{l+s+s1}{data}\PY{l+s+s1}{\PYZsq{}}\PY{p}{]}\PY{o}{.}\PY{n}{sum}\PY{p}{(}\PY{p}{)}
\end{Verbatim}


\begin{Verbatim}[commandchars=\\\{\}]
{\color{outcolor}Out[{\color{outcolor}61}]:} key
         A    15
         B    30
         C    45
         Name: data, dtype: int64
\end{Verbatim}
            
    \hypertarget{exercises}{%
\section{EXERCISES}\label{exercises}}

    EXERCISE:

In the Titanic dataset- What is the maximum Fare that was paid? And the
median?

    \begin{Verbatim}[commandchars=\\\{\}]
{\color{incolor}In [{\color{incolor}553}]:} \PY{n}{df} \PY{o}{=} \PY{n}{pd}\PY{o}{.}\PY{n}{read\PYZus{}csv}\PY{p}{(}\PY{l+s+s1}{\PYZsq{}}\PY{l+s+s1}{titanic\PYZhy{}train.csv}\PY{l+s+s1}{\PYZsq{}}\PY{p}{)}
          \PY{n}{plt}\PY{o}{.}\PY{n}{style}\PY{o}{.}\PY{n}{use}\PY{p}{(}\PY{l+s+s1}{\PYZsq{}}\PY{l+s+s1}{ggplot}\PY{l+s+s1}{\PYZsq{}}\PY{p}{)}
          \PY{k+kn}{from} \PY{n+nn}{IPython}\PY{n+nn}{.}\PY{n+nn}{core}\PY{n+nn}{.}\PY{n+nn}{interactiveshell} \PY{k}{import} \PY{n}{InteractiveShell}
          \PY{n}{InteractiveShell}\PY{o}{.}\PY{n}{ast\PYZus{}node\PYZus{}interactivity} \PY{o}{=} \PY{l+s+s2}{\PYZdq{}}\PY{l+s+s2}{all}\PY{l+s+s2}{\PYZdq{}}
          
          \PY{n+nb}{print}\PY{p}{(}\PY{n}{df}\PY{p}{[}\PY{l+s+s1}{\PYZsq{}}\PY{l+s+s1}{Fare}\PY{l+s+s1}{\PYZsq{}}\PY{p}{]}\PY{o}{.}\PY{n}{max}\PY{p}{(}\PY{p}{)}\PY{p}{)}
          \PY{n+nb}{print}\PY{p}{(}\PY{n}{df}\PY{p}{[}\PY{l+s+s1}{\PYZsq{}}\PY{l+s+s1}{Fare}\PY{l+s+s1}{\PYZsq{}}\PY{p}{]}\PY{o}{.}\PY{n}{median}\PY{p}{(}\PY{p}{)}\PY{p}{)}
\end{Verbatim}


    \begin{Verbatim}[commandchars=\\\{\}]
512.3292
14.4542

    \end{Verbatim}

    EXERCISE:

Calculate the average survival ratio for all passengers without using
the mean function (note: the `Survived' column indicates whether someone
survived (1) or not (0)).

    \begin{Verbatim}[commandchars=\\\{\}]
{\color{incolor}In [{\color{incolor}554}]:} \PY{c+c1}{\PYZsh{}df[\PYZsq{}Survived\PYZsq{}].aggregate(lambda x: x.sum() / len(x))}
          \PY{c+c1}{\PYZsh{}or}
          \PY{n+nb}{print}\PY{p}{(}\PY{n}{df}\PY{p}{[}\PY{l+s+s1}{\PYZsq{}}\PY{l+s+s1}{Survived}\PY{l+s+s1}{\PYZsq{}}\PY{p}{]}\PY{o}{.}\PY{n}{sum}\PY{p}{(}\PY{p}{)}\PY{o}{/}\PY{n+nb}{len}\PY{p}{(}\PY{n}{df}\PY{p}{)}\PY{p}{)}
\end{Verbatim}


    \begin{Verbatim}[commandchars=\\\{\}]
0.3838383838383838

    \end{Verbatim}

    EXERCISE:

Plot the age distribution of the titanic passengers

    \begin{Verbatim}[commandchars=\\\{\}]
{\color{incolor}In [{\color{incolor}555}]:} \PY{n}{df}\PY{p}{[}\PY{l+s+s1}{\PYZsq{}}\PY{l+s+s1}{Age}\PY{l+s+s1}{\PYZsq{}}\PY{p}{]}\PY{o}{.}\PY{n}{hist}\PY{p}{(}\PY{p}{)}
\end{Verbatim}


\begin{Verbatim}[commandchars=\\\{\}]
{\color{outcolor}Out[{\color{outcolor}555}]:} <matplotlib.axes.\_subplots.AxesSubplot at 0x1151c46a0>
\end{Verbatim}
            
    \begin{center}
    \adjustimage{max size={0.9\linewidth}{0.9\paperheight}}{output_101_1.png}
    \end{center}
    { \hspace*{\fill} \\}
    
    EXERCISE:

Based on the titanic data set, select all rows for male passengers and
calculate the mean age of those passengers. Do the same for the female
passengers

    \begin{Verbatim}[commandchars=\\\{\}]
{\color{incolor}In [{\color{incolor}556}]:} \PY{n+nb}{print}\PY{p}{(}\PY{n}{df}\PY{p}{[}\PY{n}{df}\PY{p}{[}\PY{l+s+s1}{\PYZsq{}}\PY{l+s+s1}{Sex}\PY{l+s+s1}{\PYZsq{}}\PY{p}{]}\PY{o}{==}\PY{l+s+s1}{\PYZsq{}}\PY{l+s+s1}{male}\PY{l+s+s1}{\PYZsq{}}\PY{p}{]}\PY{p}{[}\PY{l+s+s1}{\PYZsq{}}\PY{l+s+s1}{Age}\PY{l+s+s1}{\PYZsq{}}\PY{p}{]}\PY{o}{.}\PY{n}{mean}\PY{p}{(}\PY{p}{)}\PY{p}{)}
          \PY{n+nb}{print}\PY{p}{(}\PY{n}{df}\PY{p}{[}\PY{n}{df}\PY{p}{[}\PY{l+s+s1}{\PYZsq{}}\PY{l+s+s1}{Sex}\PY{l+s+s1}{\PYZsq{}}\PY{p}{]}\PY{o}{==}\PY{l+s+s1}{\PYZsq{}}\PY{l+s+s1}{female}\PY{l+s+s1}{\PYZsq{}}\PY{p}{]}\PY{p}{[}\PY{l+s+s1}{\PYZsq{}}\PY{l+s+s1}{Age}\PY{l+s+s1}{\PYZsq{}}\PY{p}{]}\PY{o}{.}\PY{n}{mean}\PY{p}{(}\PY{p}{)}\PY{p}{)}
\end{Verbatim}


    \begin{Verbatim}[commandchars=\\\{\}]
30.72664459161148
27.915708812260537

    \end{Verbatim}

    EXERCISE:

Based on the titanic data set, how many passengers older than 70 were on
the Titanic?

    \begin{Verbatim}[commandchars=\\\{\}]
{\color{incolor}In [{\color{incolor}557}]:} \PY{n+nb}{print}\PY{p}{(}\PY{n+nb}{len}\PY{p}{(}\PY{n}{df}\PY{p}{[}\PY{n}{df}\PY{p}{[}\PY{l+s+s1}{\PYZsq{}}\PY{l+s+s1}{Age}\PY{l+s+s1}{\PYZsq{}}\PY{p}{]}\PY{o}{\PYZgt{}}\PY{l+m+mi}{70}\PY{p}{]}\PY{p}{)}\PY{p}{)}
\end{Verbatim}


    \begin{Verbatim}[commandchars=\\\{\}]
5

    \end{Verbatim}

    EXERCISE:

Calculate the average age for each sex again, but now using groupby.

    \begin{Verbatim}[commandchars=\\\{\}]
{\color{incolor}In [{\color{incolor}558}]:} \PY{n+nb}{print}\PY{p}{(}\PY{n}{df}\PY{o}{.}\PY{n}{groupby}\PY{p}{(}\PY{l+s+s1}{\PYZsq{}}\PY{l+s+s1}{Sex}\PY{l+s+s1}{\PYZsq{}}\PY{p}{)}\PY{p}{[}\PY{p}{[}\PY{l+s+s1}{\PYZsq{}}\PY{l+s+s1}{Age}\PY{l+s+s1}{\PYZsq{}}\PY{p}{]}\PY{p}{]}\PY{o}{.}\PY{n}{mean}\PY{p}{(}\PY{p}{)}\PY{p}{)}
\end{Verbatim}


    \begin{Verbatim}[commandchars=\\\{\}]
              Age
Sex              
female  27.915709
male    30.726645

    \end{Verbatim}

    EXERCISE:

Calculate the average survival ratio for all passengers.

    \begin{Verbatim}[commandchars=\\\{\}]
{\color{incolor}In [{\color{incolor}559}]:} \PY{n+nb}{print}\PY{p}{(}\PY{n}{df}\PY{p}{[}\PY{l+s+s1}{\PYZsq{}}\PY{l+s+s1}{Survived}\PY{l+s+s1}{\PYZsq{}}\PY{p}{]}\PY{o}{.}\PY{n}{mean}\PY{p}{(}\PY{p}{)}\PY{p}{)}
\end{Verbatim}


    \begin{Verbatim}[commandchars=\\\{\}]
0.3838383838383838

    \end{Verbatim}

    EXERCISE:

What is the difference in the survival ratio between the sexes?

    \begin{Verbatim}[commandchars=\\\{\}]
{\color{incolor}In [{\color{incolor}560}]:} \PY{c+c1}{\PYZsh{} ASK ABOUT THIS Q}
          
          \PY{c+c1}{\PYZsh{}df.groupby(\PYZsq{}Sex\PYZsq{})[[\PYZsq{}Survived\PYZsq{}]].aggregate(lambda x: x.sum() / len(x))}
          \PY{c+c1}{\PYZsh{}df[df[\PYZsq{}Sex\PYZsq{}]==\PYZsq{}female\PYZsq{}][\PYZsq{}Survived\PYZsq{}].mean()/df[df[\PYZsq{}Sex\PYZsq{}]==\PYZsq{}male\PYZsq{}][\PYZsq{}Survived\PYZsq{}].mean()}
          \PY{n+nb}{print}\PY{p}{(}\PY{n}{df}\PY{p}{[}\PY{n}{df}\PY{p}{[}\PY{l+s+s1}{\PYZsq{}}\PY{l+s+s1}{Sex}\PY{l+s+s1}{\PYZsq{}}\PY{p}{]}\PY{o}{==}\PY{l+s+s1}{\PYZsq{}}\PY{l+s+s1}{female}\PY{l+s+s1}{\PYZsq{}}\PY{p}{]}\PY{p}{[}\PY{l+s+s1}{\PYZsq{}}\PY{l+s+s1}{Survived}\PY{l+s+s1}{\PYZsq{}}\PY{p}{]}\PY{o}{.}\PY{n}{mean}\PY{p}{(}\PY{p}{)} \PY{o}{\PYZhy{}} \PY{n}{df}\PY{p}{[}\PY{n}{df}\PY{p}{[}\PY{l+s+s1}{\PYZsq{}}\PY{l+s+s1}{Sex}\PY{l+s+s1}{\PYZsq{}}\PY{p}{]}\PY{o}{==}\PY{l+s+s1}{\PYZsq{}}\PY{l+s+s1}{male}\PY{l+s+s1}{\PYZsq{}}\PY{p}{]}\PY{p}{[}\PY{l+s+s1}{\PYZsq{}}\PY{l+s+s1}{Survived}\PY{l+s+s1}{\PYZsq{}}\PY{p}{]}\PY{o}{.}\PY{n}{mean}\PY{p}{(}\PY{p}{)}\PY{p}{)}
\end{Verbatim}


    \begin{Verbatim}[commandchars=\\\{\}]
0.5531300709799203

    \end{Verbatim}

    EXERCISE:

Calculate this survival ratio for all male passengers younger that 25
that belongs to Pclass 1.

    \begin{Verbatim}[commandchars=\\\{\}]
{\color{incolor}In [{\color{incolor}561}]:} \PY{n+nb}{print}\PY{p}{(}\PY{n}{df}\PY{p}{[}\PY{p}{(}\PY{n}{df}\PY{p}{[}\PY{l+s+s1}{\PYZsq{}}\PY{l+s+s1}{Sex}\PY{l+s+s1}{\PYZsq{}}\PY{p}{]}\PY{o}{==}\PY{l+s+s1}{\PYZsq{}}\PY{l+s+s1}{male}\PY{l+s+s1}{\PYZsq{}}\PY{p}{)} \PY{o}{\PYZam{}} \PY{p}{(}\PY{n}{df}\PY{p}{[}\PY{l+s+s1}{\PYZsq{}}\PY{l+s+s1}{Age}\PY{l+s+s1}{\PYZsq{}}\PY{p}{]}\PY{o}{\PYZlt{}}\PY{l+m+mi}{25}\PY{p}{)} \PY{o}{\PYZam{}} \PY{p}{(}\PY{n}{df}\PY{p}{[}\PY{l+s+s1}{\PYZsq{}}\PY{l+s+s1}{Pclass}\PY{l+s+s1}{\PYZsq{}}\PY{p}{]}\PY{o}{==}\PY{l+m+mi}{1}\PY{p}{)}\PY{p}{]}\PY{p}{[}\PY{l+s+s1}{\PYZsq{}}\PY{l+s+s1}{Survived}\PY{l+s+s1}{\PYZsq{}}\PY{p}{]}\PY{o}{.}\PY{n}{mean}\PY{p}{(}\PY{p}{)}\PY{p}{)}
\end{Verbatim}


    \begin{Verbatim}[commandchars=\\\{\}]
0.4166666666666667

    \end{Verbatim}

    EXERCISE:

Or how does it differ between the different classes? Make a bar plot
visualizing the survival ratio for the 3 Pclass classes.

    \begin{Verbatim}[commandchars=\\\{\}]
{\color{incolor}In [{\color{incolor}562}]:} \PY{c+c1}{\PYZsh{}df.groupby(\PYZsq{}Pclass\PYZsq{})[\PYZsq{}Survived\PYZsq{}].aggregate(lambda x: x.sum() / len(x)).plot(kind=\PYZsq{}bar\PYZsq{})}
          \PY{n}{df}\PY{o}{.}\PY{n}{groupby}\PY{p}{(}\PY{l+s+s1}{\PYZsq{}}\PY{l+s+s1}{Pclass}\PY{l+s+s1}{\PYZsq{}}\PY{p}{)}\PY{p}{[}\PY{l+s+s1}{\PYZsq{}}\PY{l+s+s1}{Survived}\PY{l+s+s1}{\PYZsq{}}\PY{p}{]}\PY{o}{.}\PY{n}{mean}\PY{p}{(}\PY{p}{)}\PY{o}{.}\PY{n}{plot}\PY{p}{(}\PY{n}{kind}\PY{o}{=}\PY{l+s+s1}{\PYZsq{}}\PY{l+s+s1}{bar}\PY{l+s+s1}{\PYZsq{}}\PY{p}{)}
\end{Verbatim}


\begin{Verbatim}[commandchars=\\\{\}]
{\color{outcolor}Out[{\color{outcolor}562}]:} <matplotlib.axes.\_subplots.AxesSubplot at 0x1155025f8>
\end{Verbatim}
            
    \begin{center}
    \adjustimage{max size={0.9\linewidth}{0.9\paperheight}}{output_115_1.png}
    \end{center}
    { \hspace*{\fill} \\}
    
    EXERCISE:

Make a bar plot to visualize the average Fare payed by people depending
on their age.use \texttt{pd.cut} to divide the age column.

    \begin{Verbatim}[commandchars=\\\{\}]
{\color{incolor}In [{\color{incolor}563}]:} \PY{n}{binedges} \PY{o}{=} \PY{n}{np}\PY{o}{.}\PY{n}{arange}\PY{p}{(}\PY{l+m+mi}{0}\PY{p}{,}\PY{n}{df}\PY{o}{.}\PY{n}{Age}\PY{o}{.}\PY{n}{max}\PY{p}{(}\PY{p}{)}\PY{o}{+}\PY{l+m+mi}{10}\PY{p}{,}\PY{l+m+mi}{10}\PY{p}{)}
          \PY{n}{binlabels} \PY{o}{=} \PY{p}{[}\PY{l+s+s1}{\PYZsq{}}\PY{l+s+s1}{0\PYZhy{}10}\PY{l+s+s1}{\PYZsq{}}\PY{p}{,}\PY{l+s+s1}{\PYZsq{}}\PY{l+s+s1}{11\PYZhy{}20}\PY{l+s+s1}{\PYZsq{}}\PY{p}{,}\PY{l+s+s1}{\PYZsq{}}\PY{l+s+s1}{21\PYZhy{}30}\PY{l+s+s1}{\PYZsq{}}\PY{p}{,}\PY{l+s+s1}{\PYZsq{}}\PY{l+s+s1}{31\PYZhy{}40}\PY{l+s+s1}{\PYZsq{}}\PY{p}{,}\PY{l+s+s1}{\PYZsq{}}\PY{l+s+s1}{41\PYZhy{}50}\PY{l+s+s1}{\PYZsq{}}\PY{p}{,}\PY{l+s+s1}{\PYZsq{}}\PY{l+s+s1}{51\PYZhy{}60}\PY{l+s+s1}{\PYZsq{}}\PY{p}{,}\PY{l+s+s1}{\PYZsq{}}\PY{l+s+s1}{61\PYZhy{}70}\PY{l+s+s1}{\PYZsq{}}\PY{p}{,}\PY{l+s+s1}{\PYZsq{}}\PY{l+s+s1}{71\PYZhy{}80}\PY{l+s+s1}{\PYZsq{}}\PY{p}{]}
          
          \PY{n}{df}\PY{p}{[}\PY{l+s+s1}{\PYZsq{}}\PY{l+s+s1}{age\PYZus{}bins}\PY{l+s+s1}{\PYZsq{}}\PY{p}{]} \PY{o}{=} \PY{n}{pd}\PY{o}{.}\PY{n}{cut}\PY{p}{(}\PY{n}{df}\PY{o}{.}\PY{n}{Age}\PY{p}{,}\PY{n}{bins}\PY{o}{=}\PY{n}{binedges} \PY{p}{,}\PY{n}{labels}\PY{o}{=}\PY{n}{binlabels}\PY{p}{)}
          \PY{n}{ax} \PY{o}{=} \PY{n}{df}\PY{o}{.}\PY{n}{groupby}\PY{p}{(}\PY{l+s+s1}{\PYZsq{}}\PY{l+s+s1}{age\PYZus{}bins}\PY{l+s+s1}{\PYZsq{}}\PY{p}{)}\PY{p}{[}\PY{l+s+s1}{\PYZsq{}}\PY{l+s+s1}{Fare}\PY{l+s+s1}{\PYZsq{}}\PY{p}{]}\PY{o}{.}\PY{n}{mean}\PY{p}{(}\PY{p}{)}\PY{o}{.}\PY{n}{plot}\PY{p}{(}\PY{n}{kind}\PY{o}{=}\PY{l+s+s1}{\PYZsq{}}\PY{l+s+s1}{bar}\PY{l+s+s1}{\PYZsq{}}\PY{p}{,}\PY{n}{color}\PY{o}{=}\PY{l+s+s1}{\PYZsq{}}\PY{l+s+s1}{b}\PY{l+s+s1}{\PYZsq{}}\PY{p}{)}
          \PY{n}{ax}\PY{o}{.}\PY{n}{set}\PY{p}{(}\PY{n}{xlabel}\PY{o}{=}\PY{l+s+s1}{\PYZsq{}}\PY{l+s+s1}{Age categories}\PY{l+s+s1}{\PYZsq{}}\PY{p}{,} \PY{n}{ylabel}\PY{o}{=}\PY{l+s+s1}{\PYZsq{}}\PY{l+s+s1}{Average Fare}\PY{l+s+s1}{\PYZsq{}}\PY{p}{)}
\end{Verbatim}


\begin{Verbatim}[commandchars=\\\{\}]
{\color{outcolor}Out[{\color{outcolor}563}]:} [Text(0,0.5,'Average Fare'), Text(0.5,0,'Age categories')]
\end{Verbatim}
            
    \begin{center}
    \adjustimage{max size={0.9\linewidth}{0.9\paperheight}}{output_117_1.png}
    \end{center}
    { \hspace*{\fill} \\}
    
    EXERCISE:

Add another column to the dataframe and call it `Rank'. the vaules of
this column will be the rank of the `Fare' according to the `Pclass'
column. e.g if one had the maximal Fare in Pclass 1 he gets 1 (same for
the maximal of Pclass 2) and so on

    \begin{Verbatim}[commandchars=\\\{\}]
{\color{incolor}In [{\color{incolor}564}]:} \PY{n}{df}\PY{p}{[}\PY{l+s+s1}{\PYZsq{}}\PY{l+s+s1}{Rank}\PY{l+s+s1}{\PYZsq{}}\PY{p}{]} \PY{o}{=} \PY{n}{df}\PY{o}{.}\PY{n}{groupby}\PY{p}{(}\PY{l+s+s1}{\PYZsq{}}\PY{l+s+s1}{Pclass}\PY{l+s+s1}{\PYZsq{}}\PY{p}{)}\PY{p}{[}\PY{l+s+s1}{\PYZsq{}}\PY{l+s+s1}{Fare}\PY{l+s+s1}{\PYZsq{}}\PY{p}{]}\PY{o}{.}\PY{n}{rank}\PY{p}{(}\PY{n}{ascending}\PY{o}{=}\PY{k+kc}{False}\PY{p}{)}
\end{Verbatim}


    EXERCISE:

Look at the Cabin columns and turn it into another boolean column (1/0)
indicates if the passenger has a cabin or not

    \begin{Verbatim}[commandchars=\\\{\}]
{\color{incolor}In [{\color{incolor}565}]:} \PY{n}{df}\PY{p}{[}\PY{l+s+s1}{\PYZsq{}}\PY{l+s+s1}{Cabin\PYZus{}bool}\PY{l+s+s1}{\PYZsq{}}\PY{p}{]} \PY{o}{=} \PY{n}{df}\PY{p}{[}\PY{l+s+s1}{\PYZsq{}}\PY{l+s+s1}{Cabin}\PY{l+s+s1}{\PYZsq{}}\PY{p}{]}\PY{o}{.}\PY{n}{notnull}\PY{p}{(}\PY{p}{)}
\end{Verbatim}


    EXERCISE:

Let's look at the names of the passengers. You need to create column
`Title' by extracting the title of the passenger with 5 possible values:
Mrs, Mr, Miss , Master and other, when other refers to rest of the
titles.

Hint : you can use \texttt{Regex} for the extracting step

    \begin{Verbatim}[commandchars=\\\{\}]
{\color{incolor}In [{\color{incolor}566}]:} \PY{k+kn}{import} \PY{n+nn}{re}
          
          \PY{n}{regex} \PY{o}{=} \PY{l+s+s1}{\PYZsq{}}\PY{l+s+s1}{(?:\PYZdl{}|\PYZca{}| )(Mrs|Mr|Miss|Master|Mrs.|Mr.|Miss.|Master.|mrs|mr|miss|master|mrs.|mr.|miss.|master.)(?:\PYZdl{}|\PYZca{}| )}\PY{l+s+s1}{\PYZsq{}}
          
          \PY{n}{name\PYZus{}list} \PY{o}{=} \PY{n}{df}\PY{p}{[}\PY{l+s+s1}{\PYZsq{}}\PY{l+s+s1}{Name}\PY{l+s+s1}{\PYZsq{}}\PY{p}{]}\PY{o}{.}\PY{n}{tolist}\PY{p}{(}\PY{p}{)}
          \PY{n}{title} \PY{o}{=} \PY{p}{[}\PY{p}{]}
          \PY{k}{for} \PY{n}{name} \PY{o+ow}{in} \PY{n}{name\PYZus{}list}\PY{p}{:}
              \PY{n}{t}\PY{o}{=}\PY{n}{re}\PY{o}{.}\PY{n}{findall}\PY{p}{(}\PY{n}{regex}\PY{p}{,} \PY{n}{name}\PY{p}{)}
              \PY{k}{if} \PY{o+ow}{not} \PY{n}{t}\PY{p}{:}
                  \PY{n}{title}\PY{o}{.}\PY{n}{append}\PY{p}{(}\PY{l+s+s1}{\PYZsq{}}\PY{l+s+s1}{other}\PY{l+s+s1}{\PYZsq{}}\PY{p}{)}
                  \PY{k}{continue}
              \PY{k}{for} \PY{n}{ch} \PY{o+ow}{in} \PY{n}{t}\PY{p}{:}
                  \PY{n}{new\PYZus{}t} \PY{o}{=} \PY{n}{ch}\PY{o}{.}\PY{n}{replace}\PY{p}{(}\PY{l+s+s1}{\PYZsq{}}\PY{l+s+s1}{.}\PY{l+s+s1}{\PYZsq{}}\PY{p}{,} \PY{l+s+s1}{\PYZsq{}}\PY{l+s+s1}{\PYZsq{}}\PY{p}{)}
                  \PY{n}{title}\PY{o}{.}\PY{n}{append}\PY{p}{(}\PY{n}{new\PYZus{}t}\PY{p}{)}
          \PY{n}{df}\PY{p}{[}\PY{l+s+s1}{\PYZsq{}}\PY{l+s+s1}{Title}\PY{l+s+s1}{\PYZsq{}}\PY{p}{]} \PY{o}{=} \PY{n}{title}
\end{Verbatim}


    EXERCISE:

fill the Null values of the Age colum with considerable value and divide
the Age and the Fare columns to 3 bins. Finally, drop the original
columns

    \begin{Verbatim}[commandchars=\\\{\}]
{\color{incolor}In [{\color{incolor}567}]:} \PY{c+c1}{\PYZsh{}fill the null values of the Age column with considerable(?) value}
          \PY{n}{df}\PY{o}{.}\PY{n}{Age}\PY{o}{.}\PY{n}{replace}\PY{p}{(}\PY{n}{np}\PY{o}{.}\PY{n}{NaN}\PY{p}{,} \PY{l+m+mi}{0}\PY{p}{,} \PY{n}{inplace}\PY{o}{=}\PY{k+kc}{True}\PY{p}{)} \PY{c+c1}{\PYZsh{}NOT CLEAR WHAT IS A CONSIDERABLE VALUE}
          
          \PY{c+c1}{\PYZsh{}divide the age and fare columns to 3 bins}
          \PY{c+c1}{\PYZsh{}age}
          \PY{n}{binedges} \PY{o}{=} \PY{n}{np}\PY{o}{.}\PY{n}{linspace}\PY{p}{(}\PY{n}{df}\PY{o}{.}\PY{n}{Age}\PY{o}{.}\PY{n}{min}\PY{p}{(}\PY{p}{)}\PY{p}{,} \PY{n}{df}\PY{o}{.}\PY{n}{Age}\PY{o}{.}\PY{n}{max}\PY{p}{(}\PY{p}{)}\PY{p}{,} \PY{l+m+mi}{4}\PY{p}{)}
          \PY{n}{binlabels} \PY{o}{=} \PY{p}{[}\PY{n+nb}{str}\PY{p}{(}\PY{n}{binedges}\PY{p}{[}\PY{l+m+mi}{0}\PY{p}{]}\PY{p}{)}\PY{o}{+}\PY{l+s+s1}{\PYZsq{}}\PY{l+s+s1}{\PYZhy{}}\PY{l+s+s1}{\PYZsq{}}\PY{o}{+}\PY{n+nb}{str}\PY{p}{(}\PY{n+nb}{round}\PY{p}{(}\PY{n}{binedges}\PY{p}{[}\PY{l+m+mi}{1}\PY{p}{]}\PY{p}{)}\PY{p}{)}\PY{p}{,}\PY{n+nb}{str}\PY{p}{(}\PY{n+nb}{round}\PY{p}{(}\PY{n}{binedges}\PY{p}{[}\PY{l+m+mi}{1}\PY{p}{]}\PY{p}{)}\PY{p}{)}\PY{o}{+}\PY{l+s+s1}{\PYZsq{}}\PY{l+s+s1}{\PYZhy{}}\PY{l+s+s1}{\PYZsq{}}\PY{o}{+}\PY{n+nb}{str}\PY{p}{(}\PY{n+nb}{round}\PY{p}{(}\PY{n}{binedges}\PY{p}{[}\PY{l+m+mi}{2}\PY{p}{]}\PY{p}{)}\PY{p}{)}\PY{p}{,}\PY{n+nb}{str}\PY{p}{(}\PY{n+nb}{round}\PY{p}{(}\PY{n}{binedges}\PY{p}{[}\PY{l+m+mi}{2}\PY{p}{]}\PY{p}{)}\PY{p}{)}\PY{o}{+}\PY{l+s+s1}{\PYZsq{}}\PY{l+s+s1}{\PYZhy{}}\PY{l+s+s1}{\PYZsq{}}\PY{o}{+}\PY{n+nb}{str}\PY{p}{(}\PY{n+nb}{round}\PY{p}{(}\PY{n}{binedges}\PY{p}{[}\PY{l+m+mi}{3}\PY{p}{]}\PY{p}{)}\PY{p}{)}\PY{p}{]}
          \PY{n}{df}\PY{p}{[}\PY{l+s+s1}{\PYZsq{}}\PY{l+s+s1}{age\PYZus{}3\PYZus{}bins}\PY{l+s+s1}{\PYZsq{}}\PY{p}{]} \PY{o}{=} \PY{n}{pd}\PY{o}{.}\PY{n}{cut}\PY{p}{(}\PY{n}{df}\PY{o}{.}\PY{n}{Age}\PY{p}{,}\PY{n}{bins}\PY{o}{=}\PY{n}{binedges} \PY{p}{,}\PY{n}{labels}\PY{o}{=}\PY{n}{binlabels}\PY{p}{)}
          \PY{c+c1}{\PYZsh{}fare}
          \PY{n}{binedges} \PY{o}{=} \PY{n}{np}\PY{o}{.}\PY{n}{linspace}\PY{p}{(}\PY{n}{df}\PY{o}{.}\PY{n}{Fare}\PY{o}{.}\PY{n}{min}\PY{p}{(}\PY{p}{)}\PY{p}{,} \PY{n}{df}\PY{o}{.}\PY{n}{Fare}\PY{o}{.}\PY{n}{max}\PY{p}{(}\PY{p}{)}\PY{p}{,} \PY{l+m+mi}{4}\PY{p}{)}
          \PY{n}{binlabels} \PY{o}{=} \PY{p}{[}\PY{n+nb}{str}\PY{p}{(}\PY{n}{binedges}\PY{p}{[}\PY{l+m+mi}{0}\PY{p}{]}\PY{p}{)}\PY{o}{+}\PY{l+s+s1}{\PYZsq{}}\PY{l+s+s1}{\PYZhy{}}\PY{l+s+s1}{\PYZsq{}}\PY{o}{+}\PY{n+nb}{str}\PY{p}{(}\PY{n+nb}{round}\PY{p}{(}\PY{n}{binedges}\PY{p}{[}\PY{l+m+mi}{1}\PY{p}{]}\PY{p}{)}\PY{p}{)}\PY{p}{,}\PY{n+nb}{str}\PY{p}{(}\PY{n+nb}{round}\PY{p}{(}\PY{n}{binedges}\PY{p}{[}\PY{l+m+mi}{1}\PY{p}{]}\PY{p}{)}\PY{p}{)}\PY{o}{+}\PY{l+s+s1}{\PYZsq{}}\PY{l+s+s1}{\PYZhy{}}\PY{l+s+s1}{\PYZsq{}}\PY{o}{+}\PY{n+nb}{str}\PY{p}{(}\PY{n+nb}{round}\PY{p}{(}\PY{n}{binedges}\PY{p}{[}\PY{l+m+mi}{2}\PY{p}{]}\PY{p}{)}\PY{p}{)}\PY{p}{,}\PY{n+nb}{str}\PY{p}{(}\PY{n+nb}{round}\PY{p}{(}\PY{n}{binedges}\PY{p}{[}\PY{l+m+mi}{2}\PY{p}{]}\PY{p}{)}\PY{p}{)}\PY{o}{+}\PY{l+s+s1}{\PYZsq{}}\PY{l+s+s1}{\PYZhy{}}\PY{l+s+s1}{\PYZsq{}}\PY{o}{+}\PY{n+nb}{str}\PY{p}{(}\PY{n+nb}{round}\PY{p}{(}\PY{n}{binedges}\PY{p}{[}\PY{l+m+mi}{3}\PY{p}{]}\PY{p}{)}\PY{p}{)}\PY{p}{]}
          \PY{n}{df}\PY{p}{[}\PY{l+s+s1}{\PYZsq{}}\PY{l+s+s1}{fare\PYZus{}3\PYZus{}bins}\PY{l+s+s1}{\PYZsq{}}\PY{p}{]} \PY{o}{=} \PY{n}{pd}\PY{o}{.}\PY{n}{cut}\PY{p}{(}\PY{n}{df}\PY{o}{.}\PY{n}{Fare}\PY{p}{,}\PY{n}{bins}\PY{o}{=}\PY{n}{binedges} \PY{p}{,}\PY{n}{labels}\PY{o}{=}\PY{n}{binlabels}\PY{p}{)}
          
          \PY{c+c1}{\PYZsh{}drop the original columns}
          \PY{n}{df}\PY{o}{.}\PY{n}{drop}\PY{p}{(}\PY{l+s+s1}{\PYZsq{}}\PY{l+s+s1}{Age}\PY{l+s+s1}{\PYZsq{}}\PY{p}{,} \PY{n}{axis}\PY{o}{=}\PY{l+m+mi}{1}\PY{p}{,} \PY{n}{inplace}\PY{o}{=}\PY{k+kc}{True}\PY{p}{)}
          \PY{n}{df}\PY{o}{.}\PY{n}{drop}\PY{p}{(}\PY{l+s+s1}{\PYZsq{}}\PY{l+s+s1}{Fare}\PY{l+s+s1}{\PYZsq{}}\PY{p}{,} \PY{n}{axis}\PY{o}{=}\PY{l+m+mi}{1}\PY{p}{,} \PY{n}{inplace}\PY{o}{=}\PY{k+kc}{True}\PY{p}{)}
\end{Verbatim}


    \hypertarget{now-lets-create-our-own-dataset}{%
\section{Now let's create our own
dataset}\label{now-lets-create-our-own-dataset}}

    EXERCISE:

\begin{verbatim}
Genrate a DataFrame with the following: 
\end{verbatim}

200 rows, 5 columns - the first one must be generated out of this list :
cat,dog,horse,pig,None (the python equivallent to Null in C++) and with
uniform distribution.The rest of the columns need to be generated out of
standard normal distribution The columns names are : key,A,B,C,D

    \begin{Verbatim}[commandchars=\\\{\}]
{\color{incolor}In [{\color{incolor}568}]:} \PY{c+c1}{\PYZsh{} create an empty df with column names: key, A, B, C, D}
          \PY{n}{col\PYZus{}names} \PY{o}{=}  \PY{p}{[}\PY{l+s+s1}{\PYZsq{}}\PY{l+s+s1}{key}\PY{l+s+s1}{\PYZsq{}}\PY{p}{,}\PY{l+s+s1}{\PYZsq{}}\PY{l+s+s1}{A}\PY{l+s+s1}{\PYZsq{}}\PY{p}{,} \PY{l+s+s1}{\PYZsq{}}\PY{l+s+s1}{B}\PY{l+s+s1}{\PYZsq{}}\PY{p}{,} \PY{l+s+s1}{\PYZsq{}}\PY{l+s+s1}{C}\PY{l+s+s1}{\PYZsq{}}\PY{p}{,}\PY{l+s+s1}{\PYZsq{}}\PY{l+s+s1}{D}\PY{l+s+s1}{\PYZsq{}}\PY{p}{]}
          \PY{n}{numRows} \PY{o}{=} \PY{l+m+mi}{200}
          \PY{n}{df} \PY{o}{=} \PY{n}{pd}\PY{o}{.}\PY{n}{DataFrame}\PY{p}{(}\PY{n}{index}\PY{o}{=}\PY{n+nb}{range}\PY{p}{(}\PY{n}{numRows}\PY{p}{)}\PY{p}{,}\PY{n}{columns}\PY{o}{=}\PY{n}{col\PYZus{}names}\PY{p}{)}
          
          \PY{c+c1}{\PYZsh{} fill the first column with \PYZhy{} cat, dog, horse, pig, None with uniform distribution. }
          \PY{n}{animals} \PY{o}{=} \PY{p}{[}\PY{l+s+s1}{\PYZsq{}}\PY{l+s+s1}{cat}\PY{l+s+s1}{\PYZsq{}}\PY{p}{,} \PY{l+s+s1}{\PYZsq{}}\PY{l+s+s1}{dog}\PY{l+s+s1}{\PYZsq{}}\PY{p}{,} \PY{l+s+s1}{\PYZsq{}}\PY{l+s+s1}{horse}\PY{l+s+s1}{\PYZsq{}}\PY{p}{,} \PY{l+s+s1}{\PYZsq{}}\PY{l+s+s1}{pig}\PY{l+s+s1}{\PYZsq{}}\PY{p}{,}\PY{k+kc}{None}\PY{p}{]}
          \PY{n}{df}\PY{p}{[}\PY{l+s+s1}{\PYZsq{}}\PY{l+s+s1}{key}\PY{l+s+s1}{\PYZsq{}}\PY{p}{]} \PY{o}{=} \PY{n}{np}\PY{o}{.}\PY{n}{random}\PY{o}{.}\PY{n}{choice}\PY{p}{(}\PY{n}{animals}\PY{p}{,} \PY{n}{numRows}\PY{p}{)}
          
          \PY{c+c1}{\PYZsh{} fill the rest of the columns out of standard normal distribution. }
          \PY{k}{for} \PY{n}{i} \PY{o+ow}{in} \PY{n+nb}{range}\PY{p}{(}\PY{l+m+mi}{1}\PY{p}{,}\PY{l+m+mi}{5}\PY{p}{)}\PY{p}{:} 
               \PY{n}{df}\PY{p}{[}\PY{n}{col\PYZus{}names}\PY{p}{[}\PY{n}{i}\PY{p}{]}\PY{p}{]} \PY{o}{=} \PY{n}{np}\PY{o}{.}\PY{n}{random}\PY{o}{.}\PY{n}{standard\PYZus{}normal}\PY{p}{(}\PY{n}{numRows}\PY{p}{)}
\end{Verbatim}


    EXERCISE:

Now that we have some null values we need to fill them. replace it with
the last valid value of the sorted Dataframe according to column C
(ascending). Finally reset the index

    \begin{Verbatim}[commandchars=\\\{\}]
{\color{incolor}In [{\color{incolor}569}]:} \PY{c+c1}{\PYZsh{} replace the None values with the last valid value (in column key?) of the sorted df according to column C}
          \PY{n}{df} \PY{o}{=} \PY{n}{df}\PY{o}{.}\PY{n}{sort\PYZus{}values}\PY{p}{(}\PY{l+s+s1}{\PYZsq{}}\PY{l+s+s1}{C}\PY{l+s+s1}{\PYZsq{}}\PY{p}{,} \PY{n}{ascending}\PY{o}{=}\PY{k+kc}{True}\PY{p}{)}
          
          \PY{n}{rep\PYZus{}animal} \PY{o}{=} \PY{k+kc}{None}
          \PY{k}{while} \PY{n}{rep\PYZus{}animal} \PY{o+ow}{is} \PY{k+kc}{None}\PY{p}{:}
              \PY{n}{rep\PYZus{}animal} \PY{o}{=} \PY{n}{df}\PY{o}{.}\PY{n}{loc}\PY{p}{[}\PY{n}{df}\PY{p}{[}\PY{l+s+s1}{\PYZsq{}}\PY{l+s+s1}{C}\PY{l+s+s1}{\PYZsq{}}\PY{p}{]} \PY{o}{==} \PY{n}{df}\PY{p}{[}\PY{l+s+s1}{\PYZsq{}}\PY{l+s+s1}{C}\PY{l+s+s1}{\PYZsq{}}\PY{p}{]}\PY{o}{.}\PY{n}{max}\PY{p}{(}\PY{p}{)}\PY{p}{,} \PY{l+s+s1}{\PYZsq{}}\PY{l+s+s1}{key}\PY{l+s+s1}{\PYZsq{}}\PY{p}{]}\PY{o}{.}\PY{n}{iloc}\PY{p}{[}\PY{l+m+mi}{0}\PY{p}{]}
          
          \PY{c+c1}{\PYZsh{} df[\PYZsq{}key\PYZsq{}].replace(None, rep\PYZus{}animal, inplace=True)}
          \PY{n}{df}\PY{p}{[}\PY{l+s+s1}{\PYZsq{}}\PY{l+s+s1}{key}\PY{l+s+s1}{\PYZsq{}}\PY{p}{]}\PY{o}{.}\PY{n}{fillna}\PY{p}{(}\PY{n}{value}\PY{o}{=}\PY{n}{rep\PYZus{}animal}\PY{p}{,} \PY{n}{inplace}\PY{o}{=}\PY{k+kc}{True}\PY{p}{)}
          
          \PY{c+c1}{\PYZsh{} reset the index }
          \PY{c+c1}{\PYZsh{} df.reset\PYZus{}index(drop=True)}
          \PY{n}{df}\PY{o}{.}\PY{n}{reset\PYZus{}index}\PY{p}{(}\PY{p}{)}
\end{Verbatim}


\begin{Verbatim}[commandchars=\\\{\}]
{\color{outcolor}Out[{\color{outcolor}569}]:}      index    key         A         B         C         D
          0        3    pig  1.232128 -0.315534 -2.314247 -1.073575
          1      130    dog  0.408602  0.150002 -2.282880  0.968607
          2       86    dog  0.043166  0.225721 -2.242299 -0.660823
          3      185    dog -0.581559  0.738879 -2.184273  1.336626
          ..     {\ldots}    {\ldots}       {\ldots}       {\ldots}       {\ldots}       {\ldots}
          196     32    pig -0.717012  0.126499  2.148507  0.471112
          197    146    dog  1.340235 -0.859945  2.356775  0.408570
          198    106    dog -0.770709 -2.318966  2.695508 -2.826732
          199     21  horse -0.725280  1.414844  3.198812 -0.093785
          
          [200 rows x 6 columns]
\end{Verbatim}
            
    EXERCISE:

Extract for each value in the `key' column the row with the minimal
value in `A' column and return the `D' column value of the row.

Do it again, this time to the minimal of columns `B' AND `C'.

Finally, concatenate it all to new DataFrame (df2) with the following
columns: `key',`A\_MINIMAL',`B\_MINIMAL',`C\_MINIMAL' (you should have
totally 4 rows, one for each key)

    \begin{Verbatim}[commandchars=\\\{\}]
{\color{incolor}In [{\color{incolor}627}]:} \PY{c+c1}{\PYZsh{} for each column in \PYZsq{}key\PYZsq{} extract the row+D\PYZhy{}column with minimal value in A}
          \PY{n}{a} \PY{o}{=} \PY{n}{df}\PY{o}{.}\PY{n}{loc}\PY{p}{[}\PY{n}{df}\PY{o}{.}\PY{n}{groupby}\PY{p}{(}\PY{l+s+s1}{\PYZsq{}}\PY{l+s+s1}{key}\PY{l+s+s1}{\PYZsq{}}\PY{p}{)}\PY{p}{[}\PY{l+s+s1}{\PYZsq{}}\PY{l+s+s1}{A}\PY{l+s+s1}{\PYZsq{}}\PY{p}{]}\PY{o}{.}\PY{n}{idxmin}\PY{p}{(}\PY{p}{)}\PY{p}{,}\PY{p}{[}\PY{l+s+s1}{\PYZsq{}}\PY{l+s+s1}{key}\PY{l+s+s1}{\PYZsq{}}\PY{p}{,}\PY{l+s+s1}{\PYZsq{}}\PY{l+s+s1}{D}\PY{l+s+s1}{\PYZsq{}}\PY{p}{]}\PY{p}{]}
          
          \PY{c+c1}{\PYZsh{} with minimal value in B}
          \PY{n}{b} \PY{o}{=} \PY{n}{df}\PY{o}{.}\PY{n}{loc}\PY{p}{[}\PY{n}{df}\PY{o}{.}\PY{n}{groupby}\PY{p}{(}\PY{l+s+s1}{\PYZsq{}}\PY{l+s+s1}{key}\PY{l+s+s1}{\PYZsq{}}\PY{p}{)}\PY{p}{[}\PY{l+s+s1}{\PYZsq{}}\PY{l+s+s1}{B}\PY{l+s+s1}{\PYZsq{}}\PY{p}{]}\PY{o}{.}\PY{n}{idxmin}\PY{p}{(}\PY{p}{)}\PY{p}{,}\PY{p}{[}\PY{l+s+s1}{\PYZsq{}}\PY{l+s+s1}{key}\PY{l+s+s1}{\PYZsq{}}\PY{p}{,}\PY{l+s+s1}{\PYZsq{}}\PY{l+s+s1}{D}\PY{l+s+s1}{\PYZsq{}}\PY{p}{]}\PY{p}{]}
          
          \PY{c+c1}{\PYZsh{} with minimal value in C}
          \PY{n}{c} \PY{o}{=} \PY{n}{df}\PY{o}{.}\PY{n}{loc}\PY{p}{[}\PY{n}{df}\PY{o}{.}\PY{n}{groupby}\PY{p}{(}\PY{l+s+s1}{\PYZsq{}}\PY{l+s+s1}{key}\PY{l+s+s1}{\PYZsq{}}\PY{p}{)}\PY{p}{[}\PY{l+s+s1}{\PYZsq{}}\PY{l+s+s1}{C}\PY{l+s+s1}{\PYZsq{}}\PY{p}{]}\PY{o}{.}\PY{n}{idxmin}\PY{p}{(}\PY{p}{)}\PY{p}{,}\PY{p}{[}\PY{l+s+s1}{\PYZsq{}}\PY{l+s+s1}{key}\PY{l+s+s1}{\PYZsq{}}\PY{p}{,}\PY{l+s+s1}{\PYZsq{}}\PY{l+s+s1}{D}\PY{l+s+s1}{\PYZsq{}}\PY{p}{]}\PY{p}{]}
          
          \PY{c+c1}{\PYZsh{} new df2 (concatinate all)}
          \PY{n}{col\PYZus{}names} \PY{o}{=}  \PY{p}{[}\PY{l+s+s1}{\PYZsq{}}\PY{l+s+s1}{key}\PY{l+s+s1}{\PYZsq{}}\PY{p}{,}\PY{l+s+s1}{\PYZsq{}}\PY{l+s+s1}{A\PYZus{}MINIMAL}\PY{l+s+s1}{\PYZsq{}}\PY{p}{,} \PY{l+s+s1}{\PYZsq{}}\PY{l+s+s1}{B\PYZus{}MINIMAL}\PY{l+s+s1}{\PYZsq{}}\PY{p}{,} \PY{l+s+s1}{\PYZsq{}}\PY{l+s+s1}{C\PYZus{}MINIMAL}\PY{l+s+s1}{\PYZsq{}}\PY{p}{]}
          \PY{n}{numRows} \PY{o}{=} \PY{l+m+mi}{4}
          \PY{n}{df2} \PY{o}{=} \PY{n}{pd}\PY{o}{.}\PY{n}{DataFrame}\PY{p}{(}\PY{n}{columns}\PY{o}{=}\PY{n}{col\PYZus{}names}\PY{p}{)}
          \PY{n}{df2}\PY{p}{[}\PY{l+s+s1}{\PYZsq{}}\PY{l+s+s1}{key}\PY{l+s+s1}{\PYZsq{}}\PY{p}{]} \PY{o}{=} \PY{n}{a}\PY{o}{.}\PY{n}{key}\PY{o}{.}\PY{n}{values}
          \PY{n}{df2}\PY{p}{[}\PY{l+s+s1}{\PYZsq{}}\PY{l+s+s1}{A\PYZus{}MINIMAL}\PY{l+s+s1}{\PYZsq{}}\PY{p}{]} \PY{o}{=} \PY{n}{a}\PY{o}{.}\PY{n}{D}\PY{o}{.}\PY{n}{values}
          \PY{n}{df2}\PY{p}{[}\PY{l+s+s1}{\PYZsq{}}\PY{l+s+s1}{B\PYZus{}MINIMAL}\PY{l+s+s1}{\PYZsq{}}\PY{p}{]} \PY{o}{=} \PY{n}{b}\PY{o}{.}\PY{n}{D}\PY{o}{.}\PY{n}{values}
          \PY{n}{df2}\PY{p}{[}\PY{l+s+s1}{\PYZsq{}}\PY{l+s+s1}{C\PYZus{}MINIMAL}\PY{l+s+s1}{\PYZsq{}}\PY{p}{]} \PY{o}{=} \PY{n}{c}\PY{o}{.}\PY{n}{D}\PY{o}{.}\PY{n}{values}
          
          \PY{n+nb}{print}\PY{p}{(}\PY{n}{df2}\PY{p}{)}
\end{Verbatim}


    \begin{Verbatim}[commandchars=\\\{\}]
     key  A\_MINIMAL  B\_MINIMAL  C\_MINIMAL
0    cat  -1.396084  -0.805351  -1.495014
1    dog  -1.773175  -2.826732   0.968607
2  horse   0.681624   0.815851   0.959631
3    pig  -0.857030  -1.077159  -1.073575

    \end{Verbatim}


    % Add a bibliography block to the postdoc
    
    
    
    \end{document}
